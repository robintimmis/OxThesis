\chapter{\label{ch:x-misc}Miscellaneous notes}

\minitoc

\section{ORION experiment} 
The following derivation determines the polarisation of the ORION laser pulses in the experiment and the boostes frame quantities for the PIC simulations.

I will have a whole subsection devoted to the different frames of reference of relevance and then a second one about normalised units. What follows now is the derivation of the boosted frame in which the laser is incident normally relative to the lab frame where the laser is incident obliquely.

I will try to use a consistent convention for coordinate system as much as possible.

\subsection{Frames of reference}
Other frames of reference include, HB front surface at rest frame, ablating front frame, smilei frames. 

When writing out the pistoning equation in full in thesis, include analysis in Robinson 2009 to do it for multiple ion species.



I should go over this and use third year relativity notes to formalised and make more consistent.

While some of this section may seem trivial, it is frequently miscalculated in the literature, it therefore seems of great importance to provide a full derivation.

Consider a photon incident on a plasma block at angle $\theta$ as in figure \ref{fig:miscreferenceframesboosted1d}.
% TODO: \usepackage{graphicx} required
\begin{figure}
	\centering
	\includegraphics[width=1\linewidth]{figures/misc/misc_reference_frames_boosted_1D}
	\caption{}
	\label{fig:miscreferenceframesboosted1d}
\end{figure}
A boost is applied with velocity $\mathbf{v}$ to a frame such that the photon is normally incident on the now streaming plasma at velocity $-\mathbf{v}$. The velocity transformation for the photon's velocity, $\mathbf{u}$, parallel to the boost is
\begin{equation}
	\mathbf{u}'_\parallel = \frac{\mathbf{u}_\parallel - \mathbf{v}}{1-\mathbf{u}\cdot\mathbf{v}/c^2}.
\end{equation}
Setting  $\mathbf{u}'_\parallel = 0$, it is clear that
\begin{equation}
	\mathbf{v} = \mathbf{u}_\parallel = c\sin\theta \hat{\mathbf{y}}
\end{equation}
in this geometry and 
\begin{equation}
	\gamma_\mathbf{v} = \frac{1}{\sqrt{1-\mathbf{v}^2/c^2}}=\sec\theta.
\end{equation}
Noting that since Snell's law is frame invariant, the photon remains normal as it propagates into the skin depth of the plasma, a frame in which the interaction reduces to a 1D problem has been successfully found for all $\theta < \pi/2$. Those familiar with the topic may wonder how this is possible considering the `ripples' that are observed on the plasma surface for oblique incidence. The explanation for this is of course the relativity of simultaneity. It remains to determine how do all the relevant quantities transform as such a boost is applied. Starting with an easy one: the photon's wave four-vector is 
\begin{equation}
	\mathbf{K}^\mathrm{\mu} = \left(\frac{\omega}{c},\mathbf{k}\right)
\end{equation}
and thus the freqency transforms as
\begin{equation}
	\frac{\omega}{c} = \gamma_\mathbf{v}\left(\frac{\omega'}{c}-\frac{\mathbf{v}}{c}\cdot\mathbf{k'}\right).
\end{equation}
Since $\mathbf{v}\cdot\mathbf{k'} = 0$, 
\begin{equation}\label{eq:boost_omega}
	\omega' = \omega\cos\theta .
\end{equation}
As 
\begin{equation}
	n'_\mathrm{c} = \frac{m_\mathrm{e}(\omega')^2}{4\pi e^2},
\end{equation}
\begin{equation}\label{eq:boost_nc}
	n'_\mathrm{c} =n_\mathrm{c} \cos^2\theta ,
\end{equation}
while the plasma block will be Lorentz contracted along $\hat{\mathbf{y}}$, hence the number density of electrons will increase as,
\begin{equation}
	n'_\mathrm{e} = \frac{n'_\mathrm{e}}{\cos\theta},
\end{equation}
leading to the perhaps unexpected
\begin{equation}
	\bar{n}'_\mathrm{e} = \frac{\bar{n}_\mathrm{e}}{\cos^3\theta}.
\end{equation}
Time is dilated 
\begin{equation}
	t' = \frac{t}{\cos\theta}.
\end{equation}

Consider now the more general case (I should jsut simply replace my diagram with a 3D one that incorporates this initially) where the photon's electric field is rotated out of the $x$-$y$ plane, \textit{i.e.}
\begin{equation}
	\mathbf{E} = E_0(-\cos\phi\sin\theta,\cos\phi\cos\theta,\sin\phi)
\end{equation}
and correspondingly
\begin{equation}
	\mathbf{B} = \frac{\hat{\mathbf{k}} \times \mathbf{E}}{c}= \frac{E_0}{c}(\sin\phi\sin\theta,-\sin\phi\cos\theta,\cos\phi).
\end{equation}
The Lorentz transformations for electro-magnetic fields are
\begin{equation}
	\mathbf{E}'_\parallel = \mathbf{E}_\parallel,
\end{equation}
\begin{equation}
	\mathbf{B}'_\parallel = \mathbf{B}_\parallel,
\end{equation}
\begin{equation}
	\mathbf{E}'_\perp = \gamma_\mathbf{v}(\mathbf{E}_\perp + \mathbf{v} \times \mathbf{B}),
\end{equation}
\begin{equation}
	\mathbf{B}'_\perp = \gamma_\mathbf{v}(\mathbf{B}_\perp - \mathbf{v} \times \mathbf{E}/c^2).
\end{equation}
Using the above expressions for $\mathbf{E}_\perp$ and $\mathbf{E}_\parallel$ and transforming to the boosted frame,
\begin{equation}
	\mathbf{E}' = E_0\cos\theta (0,\cos\phi,\sin\phi).
\end{equation}
As anticipated for normal incidence there is no component of the E-field normal to the surface. Conveniently, the polarisation of the incident photon is unchanged despite having components both parallel and perpendicular to the transformation and 
\begin{equation}
	|\mathbf{E}'| = |\mathbf{E}|\cos\theta.
\end{equation}
The picture can now be completed. Since
\begin{equation}
	a_0' = \frac{e|\mathbf{E}'|}{m_\mathrm{e}e\omega'}
\end{equation}
it follows that
\begin{equation}
	a_0' = a_0,
\end{equation}
\begin{equation}
	S' = \frac{S}{\cos^3\theta}.
\end{equation}

\subsection{ORION interaction geometry}
The ORION target chamber has its own defined geometry with the target located at the origin, described in figure \ref{fig:miscoriontargetchambergeometry}.
% TODO: \usepackage{graphicx} required
\begin{figure}
	\centering
	\includegraphics[width=0.5\linewidth]{figures/misc/misc_ORION_target_chamber_geometry}
	\caption{ORION target chamber geometry showing the location of the target (TCC) and OHREX spectrometer and the green (SP1) and infra-red (SP2) beamlines and their corresponding polarisations.}
	\label{fig:miscoriontargetchambergeometry}
\end{figure}
The polarisation angles are $\phi_1 = \qty{11.8}{\degree}$ and $\phi_2 = \qty{16.4}{\degree}$. Following reflection of the infra-red beam off the plasma mirror, both the green and infra-red lasers propagate in the -$\hat{\mathbf{x}}$-direction  towards the origin. The OHREX crystal is located at 
\begin{equation}
	\mathbf{r}_\mathrm{OHREX} = r_0(\cos B\cos A,-\cos B\sin A, \sin B),
\end{equation}
where $r_0 = \qty{2.4}{m}$, $A = \qty{26.82} {\degree} $ and $B = \qty{18.15}{\degree}$, setting the rotation angle of the target. This was achieved using the ORION Multi-Target-Mounts. Alignment was performed by Ed Gumbrell and no further details will be provided here on that process. 

The interaction plane is therefore defined by the vector
\begin{equation}
	\mathbf{n} = \frac{\mathbf{r}_\mathrm{OHREX}}{r_0} \times  \hat{\mathbf{x}} = (0,\sin B, \cos B\sin A).
\end{equation}
The cosine rule can be applied to determine the polarisation of the laser pulses in the interaction plane, for polarisation vector $\hat{\mathbf{E}}$,
\begin{equation}
	\frac{\mathbf{n}}{|\mathbf{n}|}\cdot\hat{\mathbf{E}} = \cos\theta,
\end{equation}
where $\theta$ defines the angle between the polarisation vector and the vector normal to the interaction plane. This corresponds to angles out of the interaction plane of 42.2 \degree for the SP1 beam (rotating anticlockwise out of the interaction plane when looking from TCC to parabola) and 19.6 \degree for the SP2 beam (rotating clockwise out of the interaction plane when looking from TCC to parabola). Again applying the cosine rule, the angle of incidence is 16\degree .


Next up: Polarisation on OHREX interaction plane.

The same method can be applied to determine the polarisation of the OHREX crystal interaction plane. The OHREX crystals have a nominal Bragg angle of 51.3\degree.

I still need to know the exact orientation of the OHREX but assuming it is vertical, the interaction plane is defined by
\begin{equation}
	\mathbf{n}_\mathrm{O} =  \frac{\mathbf{r}_\mathrm{OHREX}}{r_0} \times \hat{\mathbf{z}}= (-\sin A, -\cos A, 0),
\end{equation}
once it has been normalised.

Then again applying the cosine rule, this plane corresponds to angles out of the interaction plane of \qty{10.5}{\degree} for SP1 and \qty{58.9}{\degree} for SP2.

Is has been assumed that the non-linear RPM mechanism retains the polarisation of the incident laser pulse in the reflected harmonic beam.

Then since the OHREX crystal reflection is a linear process, we can decompose our incident beam into its polarisation constituents and consider what their combined intensity post reflection at the detector plane will be.

Also discuss the fabulous result that generally one can simply extract the results in the Smilei units and multiply by the relevant factors of the frame of interest and thus not worry too much about frame transformations.

Also check the boosted frame results against the bouchard thesis.


Once I have finished this section I must redo boosted section since I have made a mistake there and rethink a bit about optimum theta.

I must also at some point just state that a hat indicates a normalised vector.

