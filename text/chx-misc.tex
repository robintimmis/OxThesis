\chapter{\label{ch:x-misc}Miscellaneous notes}

\minitoc

\section{To do}

\begin{enumerate}
	\item Similarity theory details in appendix
	\item Add details on CPA and OPCPA
	\item Velocity transformations derivation (appendix)
	\item Ponderomotive heating mechanisms
	\item Particle merging
	\item Radiating particles and relativistic larmor radius
	\item smieli performance plot
	\item A collisionless fully ionised plasma
	\item Basic derivation of the Schwinger limit
	\item Calculating collision frequency
	\item Add some detail of vectorisation
\end{enumerate}

On the ZVP front
\begin{enumerate}
	\item a0 convention, when max and when varying in time
\end{enumerate}

\section{ORION experiment} 

\subsubsection{Four-potential transformation}
I dont understand what I was getting at here...
The field transformations can also be calculated from the four-potential transformation. Consider a laser pulse obliquely incident, angle $\theta$, it has 4-vector potential $\mathbf{A}^\mu = (0, A\sin\theta,A\cos\theta,0)$, where $A = A_0\sin(\mathbf{k}\cdot\mathbf{x} - \omega t)$ and $\mathbf{k} = (k\cos\theta,k\sin\theta,0)$. Applying the lorentz transformation, to the frame where the laser pulse is normally incident,
\begin{equation}
	\mathbf{A}'^\mu = (-\gamma\beta A\cos\theta/c,  - A\sin\theta, \gamma A\cos\theta,0) = (-A\sin\theta/c, -A\sin\theta, A,0) 
\end{equation}
since $\beta = \sin\theta$ in the positive $y$ direction and $\gamma = 1/\cos\theta$.

Therefore,
\begin{equation}
	\mathbf{E}'/\cos(\mathbf{k}\cdot\mathbf{x} - \omega t) = --\sin\theta\cos\theta k A_0\hat{\mathbf{x}} --- \sin\theta\cos\theta \omega A_0\hat{\mathbf{x}} + A_0\omega\cos\theta = A_0\omega',
\end{equation}
since $\sin(\mathbf{k}\cdot\mathbf{x} - \omega t) = \sin(x'k\cos\theta-\omega t'\cos\theta)$. Hence, while the normalised vecotr potentail amplitude is 

Do it does look at though the vector potential amplitude is unchanged by the transformation but it is more subtle, it is useful to define an $a_0' = eE/m_\mathrm{e}\omega$ to normalise the electric field intensity but note that this cannot be converted into the actual vector potential, it is just a useful construct.

Also note that whatever the transverse vector potential is in the boosted normal incidence frame is simply whatever the amplitude of the vector potential is in the oblique incidence frame.


\subsection{Condition on validity of hole boring expression}
Robinson \textit{et al} \cite{robinsonHoleboringRadiationPressure2009} consider for what case is the expression they derive for hole boring valid. The case they are interested in is what happens if the energy available for an ion to gain from crossing the pseudo-capacitor is less than the kinetic energy associated with the hole boring velocity. Their analysis applies for non-relativistic hole boring velocities and circular polarised laser pulses. This theory is now updated for the ZVP mechanism (linear polarised and relativistic ion velocities).

The so-called `piston' which leads to ion hole boring is the pseudocapacitor field. In section \ref{sec:zvp_energies_derivation}, the development of that field is discussed quantitatively. The peak electric field is
\begin{equation}
	E_\mathrm{C} = E_\mathrm{L} = \sqrt{\frac{I}{\epsilon_0 c}}
\end{equation}
and the peak displacement of electrons is 
\begin{equation}
	\Delta x = \frac{\epsilon_0 E_\mathrm{C}}{en_\mathrm{e}}.
\end{equation}

Considering instead the relativistic kinetic energy gained by an ion were it to fully cross the pseudocapacitor, following equation \ref{eq:zvp_T},
\begin{equation}
	T_i = Z_i \times \frac{1}{2}m_\mathrm{e}c^2 \frac{a^2_0}{\bar{n}_\mathrm{e}} = \frac{IZ_i}{2cn_mathrm{e}}.
\end{equation}
(The equation above needs more thinking about)

Ions are reflected provided,
\begin{equation}
	T_i > \frac{1}{2}m_iv^2_\mathrm{HB}.
\end{equation}

Hmm ok so in Vincenti, they approximate electron mass as much less than ion mass and therefore neglect in the momentum calculation. It also looks like they have not done full relativistic calculation, so I cannot yet say I have that. But carrying on the derivation using Vincenti expression for simplicity:

The hole-boring velocity as calculated by Vincenti \textit{et al} \cite{vincentiOpticalPropertiesRelativistic2014} is
\begin{equation}
	\frac{v_\mathrm{HB} }{c}= \sqrt{\frac{R\cos\theta}{2}}
\end{equation}

So come back to this section, once I have fully written out the hole boring calcualtino in full, include also the multiple ion species stuff and this condition.

The upshot of this condition is something like: require no low charge to mass ratio ions (ie v heavy ions) and fully ionisation, these conditions are satisfied in this area of study.

To arrive at that result, useful parts include:
composite mass density $\rho = \sum_i m_i n_i$, $m_i = A_i/N_\mathrm{A}$ and $A_i \approx 2Z_i$ for most low mass ions relevant in these plasmas.




\section{Thinking about the ZVP calculation}
When I run sims on oblique incidence, this is another theorem that could be interesting to test.

Consider now that the surface moves inwards at speed $c$. In a time $\Delta t$, an energy $\sim B_\mathrm{L}^2\Delta t$ is incident on the surface. If at such a point, there exists a pseudocapacitor with electric field $E_\mathrm{C} \sim n_\mathrm{e}x_\mathrm{e}$, then the work done by pushing it inwards is $\sim E_\mathrm{C} n_\mathrm{e}x_\mathrm{e} \Delta x \ sim E_\mathrm{C}^2\Delta t$, since surface moving inwards at speed $c$. Thus by conservation of energy, the reflected field is $B_\mathrm{R}^2 = B_\mathrm{L}^2 - E_\mathrm{C}^2$. Note that at max displacement this cannot possibly be the case and we do see that the surface stops moving inwards, however this could be due to a reduction in the laser pulse intensity since the peak has passed. Thus this could be a reasonable approximation of the phenomena.

Then the force equilibrium expression in the boosted frame is
\begin{equation}
	-B_\mathrm{L} - \sqrt{B_\mathrm{L}^2 - E_\mathrm{C}^2} \pm B_\mathrm{i} + E_\mathrm{C}
\end{equation}
working this through one finds,
\begin{equation}
	x_\mathrm{p} = \frac{\cos^2\theta}{kS}\frac{2(1\pm \sin\theta)}{\sin^2\theta \pm 2\sin\theta +2}
\end{equation}
\begin{equation}
	T \sim \left(\frac{\cos^2\theta}{kS}\frac{2(1\pm \sin\theta)}{\sin^2\theta \pm 2\sin\theta +2}\right)
\end{equation}
And thus now predicting an optimum for electron energy at $\theta \approx 30$\degree. That is quite different. It also looks nicer so I would like this to be right.

Another thing I still need to do is gonoskov technique to get bunch thickness, also do ZVP calculation in the exponential preplasma.


\section{Things I may want to include or random notes}
Note that ZVP does not describe the peak energies in the bunches, then JxB applies, since there are always some electrons outside of the well defined sharp boundary when the density is not high enough to impose adiabaticity. 


The phase at ejection is locked at the electromagnetic field peak of the laser pulse cycle. Diffraction around the target edge does occur and can be observed in Figure 10 but primarily not in the vicinity of the electron bunches (electron bunches that undergo Vacuum Laser Acceleration (VLA) will be ejected to the other side to that experiencing diffraction). As in the work of Thévenet \textit{et al}, not all electrons in the bunch will experience VLA, only those electrons propagating close to parallel to laser pulse, the rest that dephase do not gain further energy \cite{Thevenet2016}. Electrons that do not retain their phase will be randomly ponderomotively accelerated and decelerated across laser cycles, hence in the far field, there will be high energy attosecond electron bunches surrounded by a low energy noise of electrons. Note that the strong modulation of the reflected field for these high laser intensities has not yet been considered for VLA in reflection, and would likely limit possible accelerations, while HHG in transmission is always weaker \cite{cousens2020}, ensuring the presence of a fundamental laser pulse to perform the VLA

reword:
`Figure \ref{fig:3D_HHG}a compares the incident laser pulse to the strongly modulated reflected pulse in the 3D PIC simulation. 
\begin{figure}
	\centering
	\includegraphics[width=\textwidth]{3D_HHG.png}
	\caption{\textbf{Electric field temporal structure in 3D Particle-In-Cell (PIC) simulation with $\mathbf{a_0 = 100}$, $\mathbf{\bar{n}_\mathrm{e}}$ = 100.} a) Temporal variation of the normalised vector potential of the incident and reflected laser pulses along the polarisation axis of the incident laser pulse. The reflected pulse demonstrates attosecond radiation spikes without the need for spectral filtering. b) The spectral intensity of the reflected radiation obtained via a Fourier transform of the pulse in a). The fit is calculated following the methodology of Edwards and Mikhailova \cite{edwards2020x}: $\omega_\mathrm{b}/\omega_\mathrm{L}$ defines the cut off above which an ordinary least squares fit to $\sim n^{-p}$ yields an exponent, $p > 4/3$. Beyond the cutoff the spectrum is predicted to scale as $\sim n^{-10/3}$. The fit is a simple weighted polynomial fit to the logarithm of the data using the NumPy polyfit module.}
	\label{fig:3D_HHG}
\end{figure}
The Fourier transform of the reflected pulse is presented in Figure \ref{fig:3D_HHG}b. Due to the high intensities in these simulations, the spectrum is of the modified CSE type detailed by Edwards and Mikhailova \cite{edwards2020x}: initially the spectral intensity scales as $\sim n^{-4/3}$ up to a cut off determined by the advance time bunch width of radiating electrons after which it scales as $\sim n^{-10/3}$. Edwards and Mikhailova demonstrated that this cut off, extracted from the internal dynamics of the system can be well approximated by the point where the fit to the spectrum drops below the $\sim n^{-4/3}$ scaling, at harmonic number, $n = 12$ in this simulation. This 3D simulation result is consistent with the $n = 11.3$ determined by Edwards and Mikhailova in their most similar 1D simulation at $a_0 =100$, $\bar{n}_\mathrm{e} = 90$, $\theta = 45$\degree. It is also interesting that since their definition of the bunch width corresponds to the temporal width of the radiation spike at observation, taking the full-width-half-maximum of the CSE type spikes (between $t = 15$ and 26 fs) in Figure \ref{fig:3D_HHG}a) as the cut off harmonic for each spike gives a mean harmonic cut off of $n = 11.4$ consistent with the spectrum fit and corresponding to an average pulse duration of 292 as. Hence, the cut off can infer the attosecond pulse duration from a simple UV spectrometer measurement without the need for complex attosecond resolution diagnostics. A second cut off dependent on the peak gamma factor of radiating electrons and beyond which the spectrum decays exponentially is not captured at this simulation resolution. The deviation of the spectrum from regularly spaced harmonics is a natural consequence of the high laser pulse intensity: the non-negligible hole boring velocity (scaling linearly with the electric field strength of the laser pulse \cite{robinsonRelativisticallyCorrectHoleboring2009}) significantly lengthens the path of the reflected pulse, Doppler shifting harmonics between successive pulse cycles.'

