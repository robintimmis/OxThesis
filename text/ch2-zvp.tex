\chapter{\label{ch:2-zvp}The Zero Vector Potential Absorption Mechanism}

\minitoc

\section{Introduction}
Now is presented the Zero Vector Potential mechanism of attosecond absorption, proposed by Baeva et al \cite{baeva2011} and later developed by Savin et al \cite{savin2017, savin2019zvp}. Laser energy absorption in dense plasmas was first proposed by Wilks and Kruer \cite{wilks1997}, a ponderomotive mechanism where plasma electrons are heated directly by the laser pulse via the so-called $\mathbf{J}\times \mathbf{B}$ force.

%At some point I should briefly chat about other absorption models (I think at the end of the intro - this will also include the stuff above. - Then I will start this chapter by talking about the case where JxB does not apply. ALso in the intro specify that for an overdense plasma, the laser does not propagate and must be reflected and hence we are generally talking about a laser-plasma surface interaction and the implications thereof.) Before this point I also want to discuss preplasmas.

This thesis focuses on the so-called `post-ponderomotive' regime where the frequency of the plasma oscillations ($\omega_p \sim \sqrt{n_e}$) are greater than the $\mathbf{J}\times \mathbf{B}$ induced plasma electron oscillations at $2\omega_L$. The plasma electrons are then fast enough to compensate the ponderomotive pressure of the laser pulse with the formation of electrostatic fields between electrons and ions and so respond adiabatically to the $\mathbf{J}\times \mathbf{B}$ force. Hence plasma electrons cannot be heated directly by the laser pulse. Note that this requires a sufficiently steep density gradient around the relativistic critical density surface (where $S=1$) to shift the main interaction to a region where this condition on the overdensity is satisfied. In this case the ponderomotive pressure of the laser compresses the electrons at the front surface of the plasma and so shifts the laser-plasma surface interaction to plasma densities well beyond the relativistic critical density, leaving behind a positive space charge. This electron-ion charge separation leads to the formation of a \textit{pseudo-capacitor} electrostatic field.

So we have entered a regime of adiabaticity where the plasma skin layer is confined within a potential well consisting of the ponderomotive pressure and the Coulomb potential. Consider a relativistic linearly polarised laser pulse obliquely incident, with an angle of incidence of $\theta$, on a semi-infinite plasma, existing for $x>0$. The Hamiltonian of a single electron confined within the potential well is
\begin{equation}\label{eq:hamiltonian_general}
	\mathcal{H} = c\sqrt{m^2_ec^2 + |\mathbf{p}|^2} + e\Phi.
\end{equation}
Here the first term is the electron energy, $U$, extracted from the invariant of the relativistic 4-momentum of the electron, $\mathbf{P^\mu} = (U/c, \mathbf{p})$,
\begin{equation}
	\mathbf{P^\mu \dot P_\mu} = \frac{U^2}{c^2} - |\mathbf{p}|^2 = m^2c^2.
\end{equation}
Note that while there has been growing interest in the curvature of spacetime by relativistic lasers [cite edward here], for modern high power lasers this effect is small and not relevant for this thesis. Throughout the inner product of 4-vectors is defined with the Minkowski Metric. [find alex citation on pg 89 of thesis]

The second term of equation \ref{eq:hamiltonian_general} describes the contribution to the electron's energy from the electrostatic potential of the pseudo-capacitor. Decomposing the electron's 3-momentum into orthogonal components: $p_\mathrm{prop}$, along the laser propagation direction, $p_\mathrm{pol}$, along the polarisation axis of the laser pulse and $p_\perp$, perpendicular to both, two simplifications can be made. Firstly, by canonical conservation of transverse momentum, $p_\mathrm{pol} = eA$, where $A$ is the laser vector potential. Secondly, in the case of a $p$-polarised laser pulse (the known optimum for ZVP electron bunch generation), the forces at play confine the electron trajectory to the  $p_\mathrm{prop}$-$p_\mathrm{pol}$ plane and the interaction geometry is in essence \ac{2D}.

[include a diagram alluding to this?-it is basically since B is out of the plane and all other E fields are in the plane, also perhaps provide a foot not here to explain how incidentally this all provides a succinct explanation of why p is better?]

Explicitly, the Hamiltonian is now
\begin{equation}\label{eq:hamiltonian_specific}
	\mathcal{H} = c\sqrt{m^2_ec^2 + p^2_\mathrm{prop} + e^2A^2} + e\Phi.
\end{equation}
From equation \ref{eq:hamiltonian_specific} it is clear that should the vector potential pass through zero, one of the walls of the potential well is suppressed, allowing electrons in the in the skin layer to escape the plasma, breaking adiabaticity. The necessity of vector potential zeros for this violent reconstruction of the plasma surface led Baeva et al \cite{baeva2011} to coin the term `Zero Vector Potential' mechanism to describe this process. Indeed, while in standard calculations a laser pulse will exponentially decay within a skin layer without passing through zero, Baeva et al \cite{baeva2011} were able to demonstrate in \ac{PIC} simulations that for this regime, zeros are able to propagate through the skin layer of the plasma. The explanation for this difference in mechanics relies on a Doppler shift in the laser field due to the relativistic motion of the plasma surface, and the mathematical formalism of this process proceeds as follows.

\begin{equation}
	p_y
\end{equation}