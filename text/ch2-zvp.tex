\chapter{\label{ch:2-zvp}The Zero Vector Potential Absorption Mechanism}

\minitoc
\section{Motivations and an overview}
Throughout the history of experimental science, light has always been the primary tool of investigation and discovery. Through the creation of synchrotron radiation sources and more recently XFELs, electron bunches have been employed to create increasingly specialised light sources for the study of matter of all kinds. At SLAC, the United States' forefront electron accelerator, applications hail from many disciplines: science, medicine, industry and homeland security. At Diamond Light Source, the UK's national synchrotron science facility, studies range from novel drugs to ancient paintings. Unsurprisingly, extensive research efforts have been enlisted for the production of electron bunches of ever greater charge, energy and coherency. 

Multi-petawatt laser facilities are now available across the globe for the study of laser-plasma interactions in the ultra-relativistic regime $a_0 \gg 1$. Here, there is a novel method for high-charge electron bunch creation. Via relativistic effects, a laser pulse organises the electrons at the surface of a solid-density plasma into coherent bunches that can be ejected at high speeds. Those discussed in this thesis have properties comparable to those of forefront accelerators but on ultra-short timescales. The quality, charge and duration of the attosecond electron bunches described here would enable the study of the most fundamental interactions of our universe. 

Electron bunch formation from solid targets has received much interest in recent years \cite{kulaginCharacteristicsRelativisticElectron2009, huEnhancedDenseAttosecond2015, cantonoExtensiveStudyElectron2018, serebryakovNearsurfaceElectronAcceleration2017, zhangGiantIsolatedAttosecond2020, hornyGenerationSingleAttosecond2021, ongElectronTransportNanowire2021, kulaginSubrelativisticInfraredTerahertz2021} with some experimental evidence for attosecond electron bunches from intense laser-solid interactions \cite{linIsolatedAttosecondElectron2020, cardenasSubcycleDynamicsRelativistic2019, thevenetVacuumLaserAcceleration2016}. This interest stems partially from their ability to produce higher charge bunches at lower intensities compared to gas targets \cite{linIsolatedAttosecondElectron2020}. This chapter proposes a new mass-limited target setup to generate electron bunches of extreme charge density. These electron bunches are fully characterised in 3D \ac{PIC} simulations to compare their quality to those of existing electron bunch production methods. They are described quantitatively via an extended version of the \ac{ZVP} Mechanism and the corresponding implications for laser to plasma energy absorption are also considered including the junction of \ac{ZVP} with \ac{SF-QED} effects that will be accessed by next-generation laser facilities. The \ac{ZVP} theory has direct relevance to \ac{HHG} due to the known intrinsic link between electron bunches and the reflection mechanism \cite{cousensElectronTrajectoriesAssociated2020, savinAttosecondscaleAbsorptionExtreme2017}.

These results have excited the community to perform experiments to realise these electron bunches for the creation of ultra-bright X-ray pulses: after the recent successful campaign at the ORION laser facility, detailed in Section \ref{ch:3-orion}, details are given in Chapter \ref{ch:4-gemini} for the upcoming experiment to observe the electron bunches directly at the GEMINI PW laser at the Central Laser Facility, UK.

This Chapter is organised as follows. Section \ref{sec:zvp-intro} outlines the \ac{ZVP} mechanism including quantitative calculations of system properties. In Section \ref{sec:zvp-3Dsimulations} the first 3D PIC simulations to observe the ZVP mechanism are presented. Further PIC simulations are presented in Section \ref{sec:zvp-bunches} focusing now on the hot electron bunches produced via the mechanism, their properties, energy scalings and implications for \ac{SF-QED}. Finally, concluding remarks are given in Section \ref{sec:zvp-conclusion}.

\section{Introduction}\label{sec:zvp-intro}
Here is presented one model for the interaction between a relativistic laser pulse and a solid density plasma: the \ac{ZVP} mechanism as proposed by \textit{Baeva et al} \cite{baevaZeroVectorPotential2011} and later developed by \textit{Savin et al} \cite{savinAttosecondscaleAbsorptionExtreme2017,savinModellingLaserPlasmaInteractions2019}. Alongside the theories of the Relativistic Electron Spring \cite{gonoskovUltrarelativisticNanoplasmonicsRoute2011, gonoskovUltrarelativisticNanoplasmonicsRoute2011} and Coherent Synchrotron Emission \cite{derbruggeEnhancedRelativisticHarmonics2010}, \ac{ZVP} is a post-ponderomotive model of attosecond absorption and reflection. The model considers a quasi-static surface equilibrium state analogously to ion acceleration in the hole boring \cite{vincentiOpticalPropertiesRelativistic2014} and light sail regimes \cite{macchiTheoryLightSail2014}.

% TODO: At some point, I should briefly chat about other absorption models (I think at the end of the intro - this will also include the stuff above. - Then I will start this chapter by talking about the case where JxB does not apply. Also, in the intro specify that for an overdense plasma, the laser does not propagate and must be reflected hence we are generally talking about a laser-plasma surface interaction and the implications thereof.) Before this point, I also want to discuss preplasmas.

An explanation for the absorption of laser energy into dense plasmas was first suggested by Wilks and Kruer \cite{wilksAbsorptionUltraIntenseLaser1992}, a ponderomotive mechanism where plasma electrons are heated directly by the laser pulse via the $\mathbf{J}\times \mathbf{B}$ force. This thesis is interested in the so-called `post-ponderomotive' regime where the frequency of relativistic plasma oscillations ($\omega_\mathrm{p} \sim \sqrt{S}$) are greater than the $\mathbf{J}\times \mathbf{B}$ induced plasma electron oscillations at $2\omega_\mathrm{L}$. The plasma electrons' response is then sufficiently rapid to compensate for the ponderomotive pressure of the laser pulse with the formation of electrostatic fields between electrons and ions and so respond adiabatically to the applied $\mathbf{J}\times \mathbf{B}$ force. Hence, plasma electrons cannot be heated directly by the laser pulse. Interestingly, this condition suggests a criterion for the \ac{ZVP} regime, $S > 4$, slightly more constraining than $S>1$ as is typically stated for the \ac{ZVP} regime \cite{savinModellingLaserPlasmaInteractions2019}. The bulk plasma must have $S>1$ to prevent relativistic transparency and so ensure a laser-surface interaction. Then the ponderomotive pressure of the laser will typically compress the front surface to densities such that the overdensity condition is satisfied, provided the target is of sufficient thickness. Note that neglecting the pre-adiabatic formation phase requires a sufficiently steep density gradient around the relativistic critical density surface (where $S=1$) to shift the main interaction to a region where this condition on the overdensity is satisfied. Preplasma formation and scale length will be discussed in great detail in the later chapters. For now, assume the technology exists to sufficiently control this phenomenon. 

Provided all conditions are met, the ponderomotive pressure of the laser pulse compresses electrons at the front surface of the plasma and so shifts the laser-plasma surface interaction to plasma densities well beyond the relativistic critical density, leaving in its wake a positive space charge of ions. This electron-ion charge separation leads to the formation of a \textit{pseudo-capacitor} electrostatic field. Having entered a regime of adiabaticity, the plasma skin layer is confined within a potential well consisting of the ponderomotive pressure of the laser pulse and the Coulomb potential of the pseudocapacitor field. Thus is formed a high-density and longitudinally thin electron bunch (sometimes referred to as an electron sheath in the literature \cite{gonoskovUltrarelativisticNanoplasmonicsRoute2011}) at the plasma surface.

To understand this system, consider now a relativistic linearly polarised laser pulse obliquely incident, at an angle of incidence $\theta$, on a semi-infinite plasma, existing for $x\ge 0$ as in Figure \ref{fig:zvpforcesppol}.

\begin{figure}
	\centering
	\includegraphics[width=0.7\linewidth]{figures/zvp/zvp_forces_ppol}
	\caption[Diagram of a p-polarised laser pulse incident at angle $\theta$ specularly reflected from a solid density plasma.]{\textbf{Diagram of a p-polarised laser pulse incident at angle $\theta$ specularly reflected from a solid density plasma.} From consideration of the Lorentz force equation, it is clear that all forces and therefore all plasma particle dynamics are confined to a plane.}
	\label{fig:zvpforcesppol}
\end{figure}
The Hamiltonian of a single electron confined within the potential well \cite{goldsteinClassicalMechanics2013} is
\begin{equation}\label{eq:hamiltonian_general}
	\mathcal{H} = c\sqrt{m^2_\mathrm{e}c^2 + |\mathbf{p}|^2} - e\Phi.
\end{equation}
Here, the second term of Equation \ref{eq:hamiltonian_general} describes the contribution to the electron's energy from the electrostatic potential, $\Phi$, of the pseudo-capacitor. The first term is the electron energy, $U$, extracted from the invariant of the relativistic 4-momentum of the electron, $\mathbf{P^\mu} = (U/c, \mathbf{p})$,
\begin{equation}\label{eq:zvp-energy_invariant}
	\mathbf{P_\mu  P^\mu} = \frac{U^2}{c^2} - |\mathbf{p}|^2 = m^2_\mathrm{e}c^2.
\end{equation}
%TODO describe in abbreviations that U means total energy
Decomposing the electron's 3-momentum into orthogonal components: $p_\mathrm{prop}$, along the laser propagation direction, $p_\mathrm{pol}$, along the polarisation axis of the laser pulse and $p_\perp$, perpendicular to both, there are two simplifications to be made. Firstly, by canonical conservation of transverse momentum, $p_\mathrm{pol} = eA$, where $A$ is the laser vector potential amplitude. Secondly, in the case of a p-polarised laser pulse (the known optimum for ZVP electron bunch generation \cite{savinAttosecondscaleAbsorptionExtreme2017} and \ac{HHG} \cite{baevaTheoryHighorderHarmonic2006}), with reference to Figure \ref{fig:zvpforcesppol} and the Lorentz force law, the forces at play confine the electron trajectory to the $p_\mathrm{prop}$-$p_\mathrm{pol}$ plane and the essential interaction geometry is two-dimensional. This is provided one considers length scales smaller than the focal spot of the laser pulse on the target, such that variation of the ponderomotive pressure with the transverse shape of the laser pulse can be neglected.

% TODO potentially include a sketch of the potential well for clarity
Explicitly, the Hamiltonian can be written as
\begin{equation}\label{eq:hamiltonian_specific}
	\mathcal{H} = c\sqrt{m^2_ec^2 + p^2_\mathrm{prop} + e^2A^2} - e\Phi.
\end{equation}
Equation \ref{eq:hamiltonian_specific} describes a potential well that constrains the electrons: on one side is a longitudinal space charge $- e\Phi$ that prevents further propagation into the target while on the other, a term associated with the vector potential of the laser pulse, $ e^2A^2$. It is clear that should the vector potential pass through zero\footnote{Note that in writing $p_\mathrm{pol} = eA$ one has determined the gauge in which the zero becomes a defined quantity}, one of the potential well walls is totally suppressed, enabling electrons in the skin layer to escape the plasma, breaking adiabaticity. The necessity of vector potential zeros for this violent reconstruction of the plasma surface led Baeva to coin the term `Zero Vector Potential' mechanism to describe this process \cite{baevaZeroVectorPotential2011}. Indeed, while elementary electromagnetism tells us a laser pulse will exponentially decay within a skin layer of a plasma without passing through zero, Baeva \textit{et al} \cite{baevaZeroVectorPotential2011} demonstrated in \ac{PIC} simulations that in this non-linear regime, zeros do exist and do propagate through the skin layer. The explanation relies on a Doppler shift in the laser field due to the relativistic motion of the ablating plasma electrons, and a mathematical formalism of this process proceeds as follows.

%TODO classical interaction with a plasma and skin depth
% TODO: \textbf{**Must also discuss the impact of relativity in creating the observed phenomena, probably at the energy scaling section.**}

As the \ac{ZVP} mechanism is a relativistic phenomenon, it is absolutely essential to perform a relativistic analysis. Since all accelerated electrons travel at approximately speed $c$, surface electrons undergo similar trajectories. Acting collectively they oscillate in the laser pulse field. Consider first a transformation to the frame of reference where the laser pulse is normally incident to the plasma surface, this frame travels at velocity $\mathbf{v} = (c\sin\theta )\hat{\mathbf{y}}$ with electrons streaming at $-\mathbf{v}$. This is the Bourdier frame of reference, detailed in Appendix \ref{app:1-basics}.
Using Equation \ref{eq:zvp-energy_invariant}, $U = \gamma m_\mathrm{e} c^2$ and integrating Equation \ref{eq:intro-transverse_momentum_differential_equation} in the boosted frame noting $p_\mathrm{T}(t=0) = \gamma_\mathbf{v} m_\mathrm{e}v =m_\mathrm{e}c \sin\theta\cos\theta$,
\begin{equation}
	\gamma^2 = 1 + (a_0 + \sin\theta\cos\theta)^2 + \left(\frac{p_\mathrm{prop}}{m_\mathrm{e}c}\right)^2,
\end{equation}
where all parameters are in the boosted frame. Using $\mathbf{p} = \gamma m_\mathrm{e} \mathbf{v}$, the longitudinal velocity is
\begin{equation}\label{eq:zvp-vprop}
	v_\mathrm{prop} = \frac{\tilde{p}_\mathrm{prop}c}{\sqrt{1 + (a_0 + \sin\theta\cos\theta)^2 + \tilde{p}^2_\mathrm{prop}}},
\end{equation}
where $\tilde{p}_\mathrm{prop} = p_\mathrm{prop}/m_\mathrm{e}c$. Thus, should the transverse vector potential pass through $-\sin\theta\cos\theta$, zero for normal incidence, the surface can propagate towards the laser pulse at very close to speed $c$. Transforming back to the laboratory frame, at the peak of ablation ($\mathbf{u}\approx -c\hat{\mathbf{x}}$) and using the equations for relativistic velocity addition,
\begin{equation}\label{eq:zvp_velocityaddition1}
	\mathbf{u}'_{\|} = \frac{\mathbf{u}-\mathbf{v}}{1- \mathbf{u}\cdot\mathbf{v}/c^2},
\end{equation}
\begin{equation}\label{eq:zvp_velocityaddition2}
	\mathbf{u}'_{\perp} = \frac{\mathbf{u}_\perp}{\gamma_v(1- \mathbf{u}\cdot\mathbf{v}/c^2)},
\end{equation}
where $\gamma_v = 1/\sqrt{1-|\mathbf{v}|^2/c^2}$ \cite{steaneRelativityMadeRelatively2012}, one finds that this peak ablation at speed $\approx c$ occurs now in the specular reflection direction. Simultaneity is broken and ripples co-move along the surface with the incident laser pulse wavefronts.

Transform now to the rest frame of the ablating front. Beyond the relativistic critical density surface, the vector potential of the laser pulse decays evanescently. At the spatial centre of the laser pulse, it can be described simply by
\begin{equation}
	\mathbf{A}'_\mathrm{L}(t',r') = A'_0\cos(\omega'_\mathrm{L}t')\exp(-r'/\delta')\hat{\mathbf{r}}'_\mathrm{pol}= A'_\mathrm{L}\hat{\mathbf{r}}'_\mathrm{pol},
\end{equation}
where the primed symbols indicate that these quantities are measured in the rest frame of the expanding front. $A'_0$ is the vector potential amplitude and $\omega'_\mathrm{L}$ is the frequency of the laser pulse, $r'$ is the propagation distance of the laser into the plasma, $\delta'$ is the skin depth and $\hat{\mathbf{r}}'_\mathrm{pol}$ a unit vector defining the polarisation direction of the laser pulse. Un-primed coordinates will indicate the lab frame measurements.

% TODO PIC simulation of surface ripples?

While previous demonstrations of the existence of vector potential zeros assumed that the ablation occurs normally to the plasma surface, it is necessary to confirm that zeros are still predicted for specular ablation. Consider a p-polarised laser pulse confined to the $x$-$y$ plane incident with an angle of incidence $\theta$ on an ablating overdense plasma expanding with velocity $-v_f\hat{\mathbf{x}}$ in the lab frame, as in Figure \ref{fig:zvp_ablatingfront}.

\begin{figure}
	\centering
	\includegraphics[width=0.7\linewidth]{figures/zvp/zvp_ablating_front}
	\caption[Diagram of a $p$-polarised laser pulse incident on an ablating overdense plasma.]{\textbf{Diagram of a $p$-polarised laser pulse incident on an ablating overdense plasma.} The laser is incident obliquely at an angle of $\theta$ and is reflected specularly. The plasma ablates specularly also. Note that this process occurs on sub-laser pulse cycle timescales and it is therefore only the electrons that are ablating in this process. The interaction geometry is confined to a 2D plane.}
	\label{fig:zvp_ablatingfront}
\end{figure}
The direction of polarisation is
\begin{equation}
	\hat{\mathbf{r}}_\mathrm{pol} = \hat{\mathbf{x}}\sin{2\theta} - \hat{\mathbf{y}}\cos{2\theta}
\end{equation}
and the velocity of the rest frame of the ablating front relative to the lab frame is $-v_f\hat{\mathbf{x}}$.

Applying the Lorentz transformation to the electromagnetic 4-potential, $\mathbf{A}^\mu$, where $\Lambda_\mu^\nu$ is given by Equation \ref{eq:zvp_lorentz}, immediately from the $y$-coordinate transformation,
\begin{equation}\label{eq:zvp_lorentz_y}
	A'_\mathrm{L}\cos{2\theta'} = A_\mathrm{L}\cos{2\theta}.
\end{equation}
Applying the headlight effect for a source moving at an angle $2\theta$ to the boosted frame (a full derivation is given in Appendix \ref{sec:app_headlight}),
\begin{equation}
	\cos{(2\theta')} = \frac{\cos{(2\theta)}-\beta}{1 - \beta\cos{(2\theta)}}
\end{equation}
and rearranging Equation \ref{eq:zvp_lorentz_y}, the vector potential in the lab frame is
\begin{equation}\label{eq:zvp_labA}
	A_\mathrm{L} = \frac{1-\beta \sec{(2\theta)}}{1 - \beta\cos{(2\theta)}} A'_0\cos{(\omega'_L t')}\exp{(-r'/\delta')}.
\end{equation}
Writing the boosted frame space-time coordinates in terms of the lab frame coordinates,
\begin{equation}
	ct' = \gamma(ct-\beta x),
\end{equation}
\begin{equation}
	x' = \gamma(x-\beta ct),
\end{equation}
yields
\begin{equation}\label{eq:zvp_labAfull}
	A_\mathrm{L} =  A_0\cos{(\omega_L t - kx)}\exp{\left(-\frac{\sqrt{(x-\beta ct)^2+(y/\gamma)^2}}{\delta}\right)},
\end{equation}
where
\begin{equation}
	A_0 = \frac{1-\beta \sec{(2\theta)}}{1 - \beta\cos{(2\theta)}}A'_0,
\end{equation}
\begin{equation}
	\omega_L = \gamma \omega'_L,
\end{equation}
\begin{equation}
	k = \frac{\beta \gamma\omega'_L}{c},
\end{equation}
\begin{equation}
	\delta = \frac{\delta'}{\gamma}.
\end{equation}
The oscillatory term in Equation \ref{eq:zvp_labAfull} demonstrates the propagation of vector potential zeros within the plasma target. From the structure of this term, it would appear that these zeros are expelled from the plasma along the specular direction at a speed
\begin{equation}\label{eq:zvp-zero_v}
	v_\phi = \frac{\omega_L}{k} = \frac{c}{\beta} = -\frac{c^2}{v_\mathrm{f}}.
\end{equation}
In their original ZVP paper, Baeva laid some doubt on their version of this calculation, instead suggesting similarity theory predicts zeroes propagate at speed $c$ \cite{baevaZeroVectorPotential2011}. This is important since it directly affects the properties of the reflected radiation. It is well known that the emission of radiation occurs primarily at the point where the transverse momentum goes to zero, corresponding to the passage of the zero of the vector potential \cite{cousensElectronTrajectoriesAssociated2020} but equally that the width of the radiated pulse depends on the advance time emission point \cite{edwardsXRayEmissionEffectiveness2020}. If the zeroes moved at speed $c$, then the observed emitted pulse would be infinitely thin, producing radiation with perfect coherence in all cases. This is so in the regime detailed in Chapter \ref{ch:3-orion} where the pulse width is instead determined by the extent of radiating capability around the zero.

In the ZVP regime, where distinct bunches are formed at the plasma surface and the analysis above is valid, relevant to this chapter and Chapter \ref{ch:4-gemini}, simulations have suggested finite advanced time bunch widths decreasing rapidly with increasing laser intensity \cite{edwardsXRayEmissionEffectiveness2020}. This is precisely what is predicted by Equation \ref{eq:zvp-zero_v}. As the laser intensity increases, $v_\mathrm{f}$ naturally increases. Thus, the zeroes propagate closer to the speed of light and reduce the advanced time bunch width. The burst of radiation can be described by \ac{CSE} theory \cite{derbruggeEnhancedRelativisticHarmonics2010} or at least it's modified form \cite{edwardsXRayEmissionEffectiveness2020}, thus \ac{CSE} and\ac{ZVP} are intrinsically linked, however, while CSE focuses on reflection and the \ac{HHG} spectrum, \ac{ZVP} is concerned with laser pulse energy absorption. 

To summarise, for a sufficiently intense laser pulse, electrons at the surface of an irradiated solid target are accelerated by the laser to relativistic velocities at a fraction of a laser pulse cycle and therefore electrons both follow similar trajectories and respond adiabatically to the $\mathbf{J}\times \mathbf{B}$ force of the laser pulse. They form into a high charge density spatially thin coherent electron bunch on the front surface of the plasma but are displaced inwards from the approximately immobile ions via the ponderomotive pressure of the laser. This charge separation generates a longitudinal electrostatic pseudocapacitor field that confines electrons to a potential well on the front surface of the plasma, preventing further propagation of the electron bunch into the plasma bulk. When the zero of the vector potential passes through the electron bunch, the ponderomotive pressure instantaneously vanishes and electrons are ejected specularly from the target, co-propagating with the zeroes of the vector potential and emitting radiation.

After expulsion from the plasma, the pseudocapacitor is discharged as the electron bunch accelerates across it. Upon encountering the subsequent laser pulse peak amplitude, the bunch is then rotated back towards the plasma and launched into the bulk at high energy along the laser propagation axis (by conservation of transverse momentum in the plasma bulk), as it does so emitting coherent synchrotron radiation in transmission.

\subsection{ZVP electron bunch energies}\label{sec:zvp_energies_derivation}
In \cite{baevaZeroVectorPotential2011}, Baeva \textit{et al} propose energy scalings in \ac{1D} for an electron bunch produced in the \ac{ZVP} regime as a function of the incident laser pulse intensity and plasma density. This was extended to \ac{3D} by Savin \textit{et al} \cite{savinAttosecondscaleAbsorptionExtreme2017}. What follows is that discussion with closer consideration of both the constants of proportionality and their consequences. Note that throughout the electron bunch is treated as infinitesimally thin, as proved to be a reasonable assumption in previous work on ZVP \cite{baevaTheoryHighorderHarmonic2006, savinAttosecondscaleAbsorptionExtreme2017, savinEnergyAbsorptionLaserQED2019}. Sub-bunch dynamics have been explored in more detail by Gonoskov \textit{et al} \cite{gonoskovTheoryRelativisticRadiation2018}.

Consider again the semi-infinite block of plasma presented in Figure \ref{fig:introplasmafrequency}, normally irradiated by a laser pulse with wavelength $\lambda_\mathrm{L}$ and peak electric field, $E_\mathrm{L}$. It is now the ponderomotive pressure of the laser that displaces the electron fluid in this picture. The electron surface moves inwards until the pressure exerted by the peak instantaneous ponderomotive pressure of the laser pulse cycle,
\begin{equation}
	\mathbf{P}_\mathrm{L} = \epsilon_0 E^2_\mathrm{L} \hat{\mathbf{x}} = \epsilon_0 \left(\frac{a_0\omega_\mathrm{L}m_\mathrm{e}c}{e}\right)^2 \hat{\mathbf{x}}
\end{equation}
is equal and opposite to the pressure exerted by the pseudo-capacitor field,
\begin{equation}
	\mathbf{P}_\mathrm{C} = \frac{QE_\mathrm{C}}{\sigma} \hat{\mathbf{x}}= -\frac{(en_\mathrm{e}\Delta x)^2}{\epsilon_0}\hat{\mathbf{x}}
\end{equation} 
from Equations \ref{eq:intro_Q} and \ref{eq:intro_E}. Equating the magnitudes of $\mathbf{P}_\mathrm{L}$ and $\mathbf{P}_\mathrm{C}$, the maximum displacement inwards of electrons is
\begin{equation}\label{eq:zvp_dx}
	\Delta x \hat{\mathbf{x}} = \frac{c}{\omega_\mathrm{L}}\frac{a_0}{\bar{n}_\mathrm{e}}\hat{\mathbf{x}}  = \frac{1}{kS}\hat{\mathbf{x}},
\end{equation}
where $k$ is the wavevector of the laser pulse. Correspondingly,
\begin{equation}\label{eq:zvp_E}
	E_\mathrm{C} = \frac{en_\mathrm{e}}{\epsilon_0}\Delta x = \frac{\omega_\mathrm{L}cm_\mathrm{e}a_0}{e} = E_\mathrm{L}.
\end{equation}
Applying the results of Equations \ref{eq:zvp_dx} and \ref{eq:zvp_E}, when the ponderomotive pressure vanishes and the electron bunch is launched across the pseudo-capacitor, the relativistic kinetic energy gained by a single electron is
\begin{equation}\label{eq:zvp_T}
	T =  \int \mathbf{F}\cdot\mathrm{d}\mathbf{s} = \int^0_{\Delta x} -eE_\mathrm{C}dx = \int^0_{\Delta x}-\frac{en_\mathrm{e}x}{\epsilon_0}dx = \frac{1}{2}m_\mathrm{e}c^2\frac{a^2_0}{\bar{n}_\mathrm{e}}
\end{equation}
and an electron gamma factor,
\begin{equation}\label{eq:zvp_gamma}
	\gamma = \frac{1}{\sqrt{1-\beta^2}} = 1 + \frac{a_0^2}{2\bar{n}_\mathrm{e}}.
\end{equation}
Assuming all displaced electrons are captured by the potential well and launched as a coherent bunch, the total number of electrons in the bunch is
\begin{equation}\label{eq:zvp-Ne}
	N_\mathrm{e} = n_\mathrm{e} \sigma \Delta x = \frac{\sigma a_0 n_\mathrm{c}}{k}  = \sigma \epsilon_0 E_\mathrm{L}
\end{equation}
and hence, the total kinetic energy of the electron bunch is
\begin{equation}\label{eq:zvp_U}
	U_\mathrm{ZVP} = N_\mathrm{e} T = \frac{\sigma n_\mathrm{c}}{k}\times \frac{1}{2}m_\mathrm{e}c^2 \frac{a^3_0}{\bar{n}_\mathrm{e}}.
\end{equation}
It is now interesting to compare Equation \ref{eq:zvp_U} to the laser energy deposited upon the plasma surface and therefore consider what fraction of the laser energy can be absorbed via the \ac{ZVP} mechanism. Using $E_\mathrm{C} = E_\mathrm{L}$, Equation \ref{eq:zvp_U} can be rewritten as
\begin{equation}
	U_\mathrm{ZVP} = \frac{1}{2\omega_\mathrm{L} S}\sigma c \epsilon_0 E^2_\mathrm{L}.
\end{equation}
The expressions for energies in Equations \ref{eq:zvp_T} and \ref{eq:zvp_U} require the electron bunch to fully discharge the pseudo-capacitor before interaction with the subsequent laser pulse peak. Since the electron bunch travels at speed $\approx c$, the peak displacement (and thus the pseudo-capacitor width) must satisfy
\begin{equation}\label{eq:zvp-deltax_condition}
	\Delta x \le \frac{\lambda}{8}.
\end{equation}
Using Equation \ref{eq:zvp_dx}, it is clear Equation \ref{eq:zvp-deltax_condition} is satisfied for $ S\ge 1.3$ at normal incidence.

For the case of normal incidence, bunches are produced at a frequency of $2\omega_\mathrm{L}$, naturally following the frequency of the $\mathbf{J}\times \mathbf{B}$ force. Assuming a sinusoidal plane wave incident with surface area $\sigma$, the energy available during the pushing phase (a quarter-cycle) is
\begin{equation}
	U_\mathrm{L,1/4} = \sigma \frac{T}{4}\langle I_\mathrm{L}\rangle = \frac{2\pi}{8\omega_\mathrm{L}}\sigma c\epsilon_0E^2_\mathrm{L}.
\end{equation}
Hence,
\begin{equation}
	\eta_\mathrm{ZVP} = \frac{U_\mathrm{ZVP}}{U_\mathrm{L,1/4}} = \frac{2}{\pi S}.
\end{equation}
Interestingly, this analytical result predicts the trend observed by A. Savin \cite{savinModellingLaserPlasmaInteractions2019} in \ac{PIC} simulations both in magnitude and in scaling. Indeed, A. Savin demonstrated in numerical simulation
\begin{equation}
	\eta_\mathrm{ZVP} \sim S^{-1.000(3)},
\end{equation}
however, this result led A. Savin to conclude that increasing $S$ reduces absorption, increasing the energy in the reflected \ac{HHG} beam thus increasing high harmonic efficiency, seemingly in tension with the results of other works \cite{gonoskovUltrarelativisticNanoplasmonicsRoute2011,edwardsXRayEmissionEffectiveness2020} that suggest $S = 1$ is optima. The resolution arises from awareness of two distinct conversion efficiencies that describe the reflected harmonic spectrum: the total conversion efficiency into the full reflected beam and the conversion efficiency for individual harmonics. While the overall conversion into the reflected beam decreases for decreasing $S$, the slope of the harmonic spectrum becomes shallower and \ac{HHG} efficiency increases. Indeed, high harmonic efficiency necessitates high reflection inefficiencies due to the production of ZVP electron bunches as higher energy bunches produce more coherent reflected radiation \cite{edwardsXRayEmissionEffectiveness2020}. 
% Harmonic efficiencies will be discussed in more detail in the following Chapters, however, at currently accessible laser intensities, it would appear that X-ray harmonics are produced with greater efficiency for high $S$ \cite{pukhovRelativisticHighHarmonics2009}, a regime generally neglected in parameter scans given the computational and experimental challenges of accessing it.

Unfortunately, it is not possible to link the energy scaling derived in this Section to \ac{CSE} since the harmonic emission occurs before the \ac{ZVP} acceleration phase. In Dromey \textit{et al}, the \ac{CSE} regime was demonstrated for the first time in transmission through thin foils, however, laser pulses are now of sufficient intensity to access \ac{CSE} in reflection from solid targets. This has been identified as a more efficient approach if it can be reached due to the production of higher density and shorter duration electron bunches and therefore brighter and more coherent \ac{CSE} \cite{edwardsElectronNanobunchWidthDominatedSpectralPower2020}. 

\subsection{ZVP bunches oblique incidence scaling}
This section is inspired by ideas from the work of Gonoskov \textit{et al} \cite{gonoskovUltrarelativisticNanoplasmonicsRoute2011} and Vincenti \textit{et al} \cite{vincentiOpticalPropertiesRelativistic2014} to extend the theory of the ZVP mechanism for energy absorption to the more practical\footnote{Not only is this more feasible in experiment but has been shown to optimise HHG \cite{gonoskovUltrarelativisticNanoplasmonicsRoute2011}.} case of oblique incidence. 

Consider again the laser pulse incident on a solid density plasma existing for $x>0$ at angle $\theta$. Transforming to the frame of reference in which the laser is normally incident (quantities in this frame are indicated by the primed symbol), the electron and ion bulk plasma species stream at velocity $\mathbf{v}_\mathrm{d} = -c \sin\theta \hat{\mathbf{y}}$.  Applying the Lorentz force law along the longitudinal direction ($\hat{\mathbf{x}}$), for a displacement of the electron fluid $x'_\mathrm{e}$ (one assumes that the expression for a single electron at the surface describes the surface since all electrons follow similar trajectories), travelling at speed $\mathbf{v'}$,
\begin{equation}\label{eq:zvp_eq}
	-e(\mathbf{v'}(x'_\mathrm{e})\times (\mathbf{B}'_\mathrm{L}(x'_\mathrm{e}) + \mathbf{B}'_\mathrm{i}(x'_\mathrm{e}))\cdot \hat{\mathbf{x}} + E'_\mathrm{C}(x'_\mathrm{e}) )= 0,
\end{equation}
where the laser magnetic field,
\begin{equation}\label{eq:zvp_Bl}
	B'_\mathrm{L} = \frac{m_\mathrm{e} \omega'_\mathrm{L}a_0\sin(\omega'_\mathrm{L}t'-k'x'_\mathrm{e})}{e} \hat{\mathbf{z}}
\end{equation}
and $B_\mathrm{i}$ is the magnetic field generated by the uncompensated ion current, $\mathbf{J_\mathrm{i}} =  Zen'_\mathrm{i}(x'_\mathrm{e}) \mathbf{v}_\mathrm{d}$, where the electron fluid has been displaced. As before, from Equation \ref{eq:intro_E},
\begin{equation}\label{eq:zvp_Ec}
	E'_\mathrm{C} = \frac{en'_\mathrm{e}x'_\mathrm{e}}{\epsilon_0}.
\end{equation}
Applying Maxwell-Ampère's Law, Equation \ref{eq:intro-B1}, and noting that by symmetry there can be no variation in the magnetic field with $y'$ or $z'$, it becomes clear that
\begin{equation}\label{eq:zvp_BiJi}
	-\frac{\mathrm{d}(\mathbf{B}'_\mathrm{i})_{z'}}{\mathrm{d} x'} = \mu_0 (\mathbf{J}_\mathrm{i})_{y'}.
\end{equation}
Integrating Equation \ref{eq:zvp_BiJi} from $-\infty$ to $x'_\mathrm{e}$, noting that $\mathbf{B}_\mathrm{i}(|\mathbf{x}| = \infty) = 0$ and assuming a constant density profile $n'_\mathrm{i}$ for $x>0$ with $Zn_\mathrm{i} = n_\mathrm{e}$,
\begin{equation}\label{eq:zvp_Bi}
	\mathbf{B}'_\mathrm{i}(x'_\mathrm{e}) = \mu_0 en'_\mathrm{e}x'_\mathrm{e}c\sin(\theta)\hat{\mathbf{z}}.
\end{equation}
Using equations \ref{eq:zvp_Bl}, \ref{eq:zvp_Ec} and \ref{eq:zvp_Bi} and making the reasonable approximation that the relativistic electrons on the surface move at speed $v'_y \approx \pm c$ at peak displacement ($x'_\mathrm{e} = x'_\mathrm{p}$), \ref{eq:zvp_eq} can be written as
\begin{equation}
	-e\left(\pm c\left(\pm\frac{m_\mathrm{e}\omega'_\mathrm{L}a_0}{e} + \mu_0 en'_\mathrm{e} x'_\mathrm{p}c\sin\theta\right)+\frac{en'_\mathrm{e}x'_\mathrm{p}}{\epsilon_0}\right) = 0.
\end{equation}
For the laser to be in the pushing phase, the first term must be negative, corresponding to $\mathbf{v'}$ and $\mathbf{B}'_\mathrm{L}$ having the opposite sign, hence,
\begin{equation}
	c\left(-\frac{m_\mathrm{e}\omega'_\mathrm{L}a_0}{e} \pm \mu_0 en'_\mathrm{e} x'_\mathrm{p}c\sin\theta\right)+\frac{en'_\mathrm{e}x'_\mathrm{p}}{\epsilon_0} = 0,
\end{equation}
where here the $\pm$ tracks the sign of $\mathbf{v}'$. After some manipulation, one arrives at
\begin{equation}
	x'_\mathrm{p} = \frac{1}{k'S' (1\pm \sin\theta)}.
\end{equation}
Transforming back to the lab frame,
\begin{equation}
	x_\mathrm{p} = \frac{\cos^2\theta}{kS(1\pm \sin\theta)}.
\end{equation}
% TODO here include the plot of two foci out of phase showing the suppression and enhancement based on angle.
Convincingly, this reduces to Equation \ref{eq:zvp_dx} for $\theta =0$ and predicts the suppression and enhancement of the two surface oscillations per laser pulse cycle. Explicitly, for a laser pulse propagating at $y = x\tan\theta$, the peak displacement of the electron surface is enhanced for $\mathbf{B}_\mathrm{L}$ in the $+\hat{\mathbf{z}}$-direction and suppressed for $\mathbf{B}_\mathrm{L}$ in the $-\hat{\mathbf{z}}$-direction.

Again, integrating to find the work done as the electron bunch accelerates across the pseudocapacitor,
\begin{equation}\label{eq:zvp_Tzvp_theta}
	T (\theta)=  \int \mathbf{F}\cdot\mathrm{d}\mathbf{s} = \frac{1}{2}m_\mathrm{e}c^2\frac{a^2_0}{\bar{n}_\mathrm{e}}\frac{\cos^4\theta}{(1\pm \sin\theta)^2}.
\end{equation}
While it was to be expected that a component of the electric field acting into or out of the plasma would change the magnitude of the displacement, this result suggests that increasing the angle of incidence can increase the electron energy gain in the enhanced cycle more than the decrease in the suppressed cycle. Note that this is to be expected as it is known that HHG efficiency is improved for non-zero angle of incidence \cite{gonoskovUltrarelativisticNanoplasmonicsRoute2011, edwardsXRayEmissionEffectiveness2020}. It would be interesting to explore whether the presence of an external magnetic field could be applied to mimic the effect of oblique incidence by replacing the magnetic field due to the uncompensated ion current.

While this model would suggest an optimal angle for electron energy and therefore of \ac{HHG} of $\pi/2$, if $\theta > \pi/4$, then, if the relativistic electron bunch is travelling at $c$ along the specular reflection direction, the subsequent laser peak amplitude will never `catch up' with the electron bunch, and electrons can escape, generating high charge electron bunches in reflection as observed in experiment by Lin \textit{et al }\cite{linIsolatedAttosecondElectron2020}, but reducing the efficiency of HHG.

Finally, the total bunch energy as a function of $\theta$,
\begin{equation}\label{eq:zvp_Uzvp_theta}
	U_\mathrm{ZVP}(\theta) = \frac{\sigma n_\mathrm{c}}{k}\times \frac{1}{2}m_\mathrm{e}c^2 \frac{a^3_0}{\bar{n}_\mathrm{e}}\frac{\cos^6\theta}{(1\pm \sin\theta)^3}.
\end{equation}
As anticipated, oblique incidence can increase laser absorption efficiency for higher $S$ plasmas which are currently more easily accessed compared to the more optimal low $S$ plasmas.

\subsection{Defining characteristics of the ZVP mechanism}\label{sec:zvp-characteristics}
In her original paper on the ZVP mechanism, T. Baeva \textit{et al} \cite{baevaZeroVectorPotential2011} outlined 6 defining characteristics of the \ac{ZVP} mechanism, namely,
\begin{enumerate}
	\item The existence of vector potential zeros moving through the skin layer in the laboratory frame;
	\item The existence of zeroes in the incident laser pulse vector potential required for the formation of fast electron bunches;
	\item The generation of fast electron bunches with ultra-short temporal duration;
	\item That such fast electron bunches follow the energy scalings of equations \ref{eq:zvp_T} and total energy \ref{eq:zvp_U};
	\item Injection of the fast electron bunches is along the propagation axis of the laser pulse;
	\item There must be an intrinsic link between the fast electron bunches and coherent \ac{HHG};
\end{enumerate}
with the moving zeros within the skin layer being the key delineator between this post-ponderomotive regime of laser pulse energy absorption and other proposed mechanisms. While such observational requirements are far beyond the reaches of current experimental capacities, numerical simulations in both 1- \cite{baevaZeroVectorPotential2011} and 2-dimensions \cite{savinAttosecondscaleAbsorptionExtreme2017} have confirmed the above points. Now is presented the first \ac{3D} simulations attempting to demonstrate these criteria.

\section{Numerical simulations of the ZVP mechanism}
\subsection{The ZVP mechanism in 3D3V}\label{sec:zvp-3Dsimulations}
3D simulation results are presented in Figure \ref{fig:zvp3d} alongside comparison to an equivalent 2D simulation.
\begin{figure}
	\centering
	\includegraphics[width=0.9\linewidth]{figures/zvp/zvp_3D}
	\caption[The numerical simulation using a 3D \ac{PIC} code of the \ac{ZVP} mechanism.]{\textbf{Simulation results from a 3D \ac{PIC} simulation of the \ac{ZVP} mechanism.} a) The initialised electron number density. b) The electron number density several cycles later, the plasma bulk is intact, however, there is evidence of instabilities and electron bunches propagating through and around the plasma. c) The electron kinetic energy density at the same timestep. Note that the scale has been clipped to enable observation of both electron bunches propagating through and around the plasma bulk. Significantly higher energy density, corresponding to a higher charge density and attosecond duration is observed for the electron bunches propagating around the bulk. d-f) Plots clipped through $z=\lambda_\mathrm{L}/2$ for a-c) respectively to access the internal structure of the plasma bulk. The accompanying plots for figures e) and f) are equivalent 2D PIC simulations demonstrating excellent agreement.}
	\label{fig:zvp3d}
\end{figure}
Simulation parameters are given in Table \ref{tab:parameters}, such parameters are compatible with the 10 PW ELI-NP state-of-the-art short pulse laser facility \cite{tanakaCurrentStatusHighlights2020} incident on foam targets.
\begin{table}[]
	\begin{center}
		\begin{tabular}{lll}
			\hline \hline
			\multicolumn{3}{c}{Laser (3D, normal incidence)}   \\
			Parameters                                        & Real                                 & Smilei                         \\ \hline
			Wavelength, $\lambda$ (nm)                        & 1060                                 & 2$\pi$                      \\
			Angular frequency, $\omega_\mathrm{L}$ (fs$^{-1}$)         & 1.8                                  & 1                           \\
			Beam waist, $w_\mathrm{L}$ (nm)                            & 6$\lambda$                           & 12$\pi$                     \\
			Focal point, ($f_x$, $f_y$, $f_z$) (nm)                  & (0.5$\lambda$, 5$\lambda$, $0.5\lambda$)             & ($\pi$, 10$\pi$, $\pi$)           \vspace{0.25cm}\\ 
			Spatial envelope, $E_i$, $i = y,z$                           & \multicolumn{2}{l}{$E_i \sim e^{-(i-i_f)^2/w_\mathrm{L}^2}$}                \\
			Temporal envelope, $E_t$                          & \multicolumn{2}{l}{$E_t \sim e^{-(t-4\lambda/c)^2/((4\lambda/3c)^2\ln 2)}$} \vspace{0.15cm}\\ \hline \hline
			\multicolumn{3}{c}{Simulation box}   \\ \hline
			Size, $x \times y\times z$ (nm)                           & $2\lambda \times 9\lambda \times \lambda$          & $4\pi \times 18\pi \times 2\pi$         \\
			Sim length (fs)                                   & 35.22                                & 20$\pi$                     \\
			Spatial resolution, $\Delta x$ (nm)               & $\lambda/128$ = 8.28                 & 0.0491                      \\
			Temporal resolution, $\Delta t$ (as)              & $\Delta x/11c$ = 2.51               & 0.00446                   \vspace{0.15cm}  \\ \hline \hline
			\multicolumn{3}{c}{Collisionless, pre-ionised randomly-initialised aluminium plasma}                                               \\ \hline
			Electron $x$ profile, $n(x)$                             & \multicolumn{2}{l}{$\begin{cases}
					n_\mathrm{e} \text{ for $2\lambda \le x \le 3\lambda$,}\\
					n_\mathrm{e}e^{(x-2\lambda)/0.2\lambda} \\
					\text{\hspace{0.5cm}for $x \le 2\lambda$.}\\
				\end{cases}$}                                               \\
			Electron $y$ profile, $n(y)$                              & \multicolumn{2}{l}{$\begin{cases}
					1 \text{ for $2\lambda \le y \le 8\lambda$,}\\
					0 \text{ otherwise.}\\
				\end{cases}$}                                               \\
			Electron $z$ profile, $n(z)$                              & \multicolumn{2}{l}{$\begin{cases}
					1 \text{ for $0.125\lambda \le y \le 0.875\lambda$,}\\
					0 \text{ otherwise.}\\
				\end{cases}$}                                               \\
			Ion profile, $n_\mathrm{i}(x,y,z)$                                & \multicolumn{2}{l}{$n_\mathrm{i} = n(x)n(y)n(z)/13$}                                                    \\
			Macro-electrons per cell                       & \multicolumn{2}{l}{729}                                                               \\
			Macro-ions per cell                               & 8                                   &  \\
			Ion temperature, $T_\mathrm{i}$ (keV)              &     0             &  0                         \\
			Electron temperature, $T_\mathrm{e}$ (keV)              &     10             &  0.02                          \\ \hline \hline
			\multicolumn{3}{c}{Stability criteria}                                               \\ \hline
			$\lambda_\mathrm{D}/\Delta x$                               &0.288                                   &  \\
			$1/\Delta t \omega_\mathrm{p}$                             &24.4                                   &  \\
			$\Delta x/c\Delta t$                             &   11                               &  \\
			Macro-particles in the Debye sphere                    &210                                  &  \vspace{0.15cm}\\ \hline \hline 
			
		\end{tabular}
	\end{center}
	\caption{\label{tab:parameters} \textbf{Simulation parameters in both real and normalised Smilei simulation units for the 3D3V simulations.} }
\end{table}
Figure \ref{fig:zvp3d}c) clearly reveals the existence of high energy density electron bunches propagating through the plasma bulk in the direction of the laser pulse. Note that this ZVP criterion is a direct consequence of conservation of transverse momentum inside the plasma bulk where the laser fields cannot propagate. Figure \ref{fig:zvp3d}b) shows these bunches escape to the rear of the bulk but lose energy as they do so. Looking now at Figure \ref{fig:zvp3d}e) and the internal structure of the plasma bulk, these bunches drive two-stream and filamentation instabilities \cite{bretMultidimensionalElectronBeamplasma2010}. The bulk propagating electron bunches are accompanied by higher density electron bunches to either side of the plasma block with the side of emission alternating every half laser pulse cycle with the $\mathbf{J}\times\mathbf{B}$ force. 

\subsubsection{Considerations and convergence of 3D simulations}
The plasma specifications were chosen to minimise computational load while ensuring numerical convergence, requiring over 100 billion macroparticles. The electron temperature is raised significantly higher than that which would be expected in such a laser-plasma system so as to resolve the Debye length. Possible plasma temperatures are calculated using 1D HYADES simulations in the following Chapter. While this temperature is unphysical and will lead to some small plasma expansion over the course of the simulation (negligible on these timescales), the temperature remains small compared to the energy imparted to the electron bunches by the laser pulse. The striking similarity between the 2 and 3D simulation results is a natural consequence of the 2D nature of the interaction geometry. It is, however, still reassuring to show that previous work withstands the test of reality's geometry. The simple elegance of the ZVP mechanism is not lost in the chaos. 

For the 3D simulations only, particles were initialised randomly not regularly to avoid numerical error. Regularly initialised plasma blocks in 3D simulations rapidly blow apart due to spurious large amplitude fields generated on the large plasma surfaces due to macroscopic electron-ion charge separation at initialisation. 2D simulations with randomly initialised particles tend to produce nano-structures reminiscent of the cosmic web. The absence of these structures in 3D is telling. Unsurprisingly this error can be reduced by increasing the plasma temperature.

The longitudinal thickness of the target does not impact the interaction and is chosen for computational efficiency, indeed it is standard to consider such targets of thickness $\ge \lambda_\mathrm{L}$ as bulk targets \cite{dollarEnhancedLaserAbsorption2017}, however, for sufficiently long pulse durations, the effect of hole boring necessitates thicker targets to make this approximation as is necessary in the following Chapter.

The initial 3D simulation parameters were chosen to be consistent with previous work on the \ac{ZVP} mechanism, however, such simulations are cumbersome. In order to query the defining characteristics outlined by Baeva, a lower-resolution simulation was performed with similar parameters to the initial simulation. Comparisons between the simulation outputs are made in Figure \ref{fig:zvp3dcomparelowres} in Appendix \ref{sec:app-pic_convergence3D}. Good convergence is qualitatively demonstrated by the presence of characteristic features of the ZVP mechanism. While the instabilities are similar in structure, the change in seeding changes their exact positions. As instabilities are not the focus of this thesis this variation is of no cause for concern.

\subsubsection{Confirmation of ZVP in 3D}
Figure \ref{fig:zvppropagatingzero} tracks the transverse momentum distribution along the polarisation axis of the laser pulse of an electron bunch during its ablative journey, clearly demonstrating the existence of a singular zero of the vector potential propagating through the electron bunch and away from the plasma bulk.
\begin{figure}
	\centering
	\includegraphics[width=0.7\linewidth]{figures/zvp/zvp_propagating_zero}
	\caption[Propagation of a vector potential zero through the ablating ZVP electron bunch.]{\textbf{Propagation of a vector potential zero through the ablating ZVP electron bunch via the proxy of transverse momentum conservation.} The zero of the vector potential exists where the transverse momentum is macroscopically zero. Each line represents a timestep with time increasing with decreasing colour. With increasing time the bunch and the zero move in the -$\hat{\mathbf{x}}$-direction with the zero overtaking the electron bunch. The inset tracks the position of the zero of the vector potential with time.}
	\label{fig:zvppropagatingzero}
\end{figure}
% TODO the inset of \ref{fig:zvppropagatingzero}  has the wrong x-axis, code updated, just need to extract the plot and replace it. (plot at bunches/bunches6/bunches_3D/9/3D_analysis.ipynb)
The zero propagates at a speed of $1.4c$, corresponding to a surface velocity of $0.71c$. In this simulation, the zero propagates through the bunch over a few nanometres and before crossing the pseudocapacitor. Since Gonoskov \textit{et al} have determined a scaling for electron bunch width \cite{gonoskovUltrarelativisticNanoplasmonicsRoute2011}, calculation of the advance time bunch width, and correspondingly the duration of the attosecond pulse, only requires the calculation of the surface velocity. It is unfortunate that this emission occurs before the ZVP phase.

Further simulation results a presented in Figure \ref{fig:zvp3ddynamics}.
\begin{figure}
	\centering
	\includegraphics[width=1\linewidth]{figures/zvp/zvp_3D_dynamics}
	\caption[Electron dynamics in 3D PIC simulation for both linear and circularly polarised relativistic laser pulses.]{Electron dynamics in a 3D PIC simulation. a) and b) Electron energy density for linearly and circularly polarised laser pulses respectively. c) and d) Electron longitudinal momentum for linearly and circularly polarised laser pulses respectively.  e) Electron density at the plasma surface streaked in time. f) Electron energy density at the plasma surface streaked in time.}
	\label{fig:zvp3ddynamics}
\end{figure}
In Figures \ref{fig:zvp3ddynamics}a)-d), comparisons are made of the electron energy distributions for simulations with zeros present in the vector potential (linearly polarised) and without (circularly polarised). Evidently, zeroes are required for the formation of high energy, short duration attosecond bunches. From Figure \ref{fig:zvp3ddynamics}a) it is clear that electron bunches are injected, propagating through the plasma bulk in the laser propagation direction. Figure \ref{fig:zvp3ddynamics}c) demonstrates the quasi-monoenergeticity of the high-energy electron bunches as initially identified in 1D by Baeva. Although the shape in the phase space is more complex in 3D, the attosecond duration at a given energy is retained. Figures \ref{fig:zvp3ddynamics}e) and f) describe the surface dynamics in the linearly polarised case. One can observe the high-density bunches on the front surface with the peak in energy density occuring after acceleration across the pseudocapacitor.

Finally, Figure \ref{fig:zvphhgbeamfourier} compares the spectra of the reflected light in the presence and absence of zeros of the vector potential.
\begin{figure}
	\centering
	\includegraphics[width=0.7\linewidth]{figures/zvp/zvp_hhg_beam_fourier}
	\caption[The Fourier transform of the reflected laser pulse in 3D PIC simulations.]{\textbf{The Fourier transform of the reflected laser pulse in 3D PIC simulations both with and without zeroes in the vector potential.}}
	\label{fig:zvphhgbeamfourier}
\end{figure}
Individual harmonics are not resolved due to blueshifts between successive pulse cycles from considerable hole boring, a phenomenon discussed in great detail in the following Chapter. The spectrum from the circularly polarised light is typically over two orders of magnitude below that of linearly polarised light. Thus, all defining characteristics of the ZVP mechanism have been identified in 3D simulations with the notable exception of the energy scalings. Next-generation supercomputers will be required to perform such large scale parameter scans.
% TODO: \textbf{**To do: these dynamics plots need fixing. Should add the times, also e and f have distance instead of time on the y axis, also need to convert from timstep to time. D) needs the scale changing. also, the x-axis needs to be converted to distance. Also, do I need to change the colour of that FFT graph**}
% TODO Add details about CSE and plot from reviewers' comments including Ey field shape. Also notes in the misc chapter about VLA etc. 

\section{The ZVP electron bunch}\label{sec:zvp-bunches}
Following the excellent agreement demonstrated between 2 and 3D PIC simulations, the remainder of this Chapter explores ZVP in 2D. Previous work on the ZVP mechanism has highlighted the high energy and short duration of electron bunches, what follows is a full characterisation of their properties. A ZVP electron bunch is an electron bunch produced via the ZVP mechanism. Once produced and accelerated across the pseudocapacitor field, it is launched back in the laser propagation direction. While the bunch has no spatial separation between energies when propagating with the zero of the vector potential, the turning point of the electrons is longitudinal momentum dependent due to the Coulomb attraction of the ions after overshooting the pseudocapacitor field. Baeva \textit{et al} showed that the electron bunch has a quasi-monoenergetic spectrum: the electron bunch is now of attosecond duration in the spectro-temporal domain and there is a one-to-one relationship between energy and position with the higher energies trailing the lower energies. The full bunch is confined to 130 as while a single energy is confined to 5 as. If, however, the plasma bulk is transversely mass-limited relative to the laser spot size, when rotated back towards the plasma block, some of the electron bunch will overshoot and escape the potential well without significant chirping of the bunch as can be seen in Figure \ref{fig:zvp3d}. Such electron bunches retain their high charge density and ultra-short duration with the tradeoff of increased divergence. Note that electron bunch properties are imprinted in the transmitted harmonics: high divergence but attosecond duration.

ZVP electron bunches can therefore be placed into two categories: ultra-high charge, ultra-short duration electron bunches from mass-limited targets, hereafter labelled mass-limited electron bunches, of interest due to their unique properties, and bulk propagating bunches, hereafter labelled bulk bunches, which have lower charge densities, are imprinted with instabilities and are instead of interest due to their connection to energy absorption and reflection in this post-ponderomotive regime. To investigate these two bunch types further, 2D PIC simulations were performed.
% TODO add parameters to the Appendix 

\subsubsection{Attosecond nano-Coulomb mass-limited electron bunches}
The plots of Figure \ref{fig:zvptypicalbunch} describe a typical mass-limited ZVP electron bunch qualitatively.
\begin{figure}
	\centering
	\includegraphics[width=1\linewidth]{figures/zvp/zvp_typical_bunch}
	\caption[2D PIC simulation results qualitatively describing typical mass-limited ZVP electron bunch structure.]{\textbf{2D PIC simulation results qualitatively describing typical mass-limited ZVP electron bunch structure.} a) Electron energy density for the full simulation window, corresponds to Figure \ref{fig:zvp3d}f). The box highlights the bunch presented in the following plots. b) Electron number density of the electron bunch. c) Electron energy density of the electron bunch, the colour bar scale has been increased compared to Figure a) to observe the internal structure. d) The mean electron energy across the electron bunch, suggesting a position-dependent energy or quasi-monoenergetic nature to the electron bunch \cite{baevaZeroVectorPotential2011}. Cells with no macroparticles are black. e) The transverse phase space in the 2D simulation plane. The ellipse describes the calculated emittance. The skew of the ellipse is a consequence of a low-density tail on the phase space beyond the bottom left corner. f) This plot was extracted from the equivalent 3D simulation and describes the transverse emittance in the $z$-direction. Again the ellipse marks the emittance. The relatively well-defined border to the phase space and the mild tilt (indicating only mild divergence) are direct consequences of the 2D nature of the interaction.}
	\label{fig:zvptypicalbunch}
\end{figure}
The electron bunch under interrogation is ultra-relativistic with a mean energy of \qty{51\pm 11}{MeV} and a duration of 35 as. It propagates at an angle of -0.393 rad relative to the laser propagation direction, \textit{i.e.} the $x$-axis, and has a transverse geometric emittance in the simulation plane (the $x$-$y$ plane) of \qty{35 \pm 7}{nm.rad}. The calculation of the transverse geometric emittance, a measure of the quality of the electron beam, is given in Appendix \ref{app:1-basics-transverse_emittance}. Note that while the bunch does not propagate in the laser propagation direction, this does not mean it must be rejected under consideration of the ZVP bunch conditions. We must expand the definition of a ZVP electron bunch. In the mass-limited case, a bunch must propagate at some angle to the laser due to conservation of canonical momentum while it remains in the thrall of the laser pulse. For an equivalent bunch in a corresponding 3D simulation, the transverse geometric emittance in the $z$ plane is \qty{7.4\pm 1}{nm.rad}. This electron bunch has a total charge of 0.35 nC for a slab of plasma of thickness $0.75\lambda$ in the $z$-direction. Note again the two-dimensional nature of the interaction geometry. Electrons a distance less than twice the relativistic Larmor radius, 
\begin{equation}\label{eq:zvp_rL}
	r_\mathrm{L} = \frac{\gamma m_\mathrm{e}v}{e|\mathbf{B}|}
\end{equation}
will escape to the side of the target when rotated back towards the plasma. Here, $\gamma$ and $v$ correspond to the electron velocity. The total number of electrons in the mass-limited bunch is
\begin{equation}\label{eq:zvp_N3}
	N = 2 n_\mathrm{e} r_\mathrm{l} L_z \Delta x,
\end{equation}
where $L_z$ is the width of the plasma in the $z$-direction. Combining Equations \ref{eq:zvp_rL} and \ref{eq:zvp_N3} with \ref{eq:zvp_dx} and \ref{eq:zvp_gamma} and taking $v \approx c$ for the ultra-relativistic electron bunch, 
\begin{equation}\label{eq:zvp_N}
	N \approx 2 \gamma n_\mathrm{c} \frac{L_z}{k^2}.
\end{equation}

Equation \ref{eq:zvp_N} can be rewritten in terms of fundamental constants as
\begin{equation}\label{eq:zvp_N2}
	N \approx 2 (1+0.5\frac{a^2_0}{\bar{n}_\mathrm{e}}) L_z \frac{m_\mathrm{e}\epsilon_0c^2}{e^2}.
\end{equation}
For these simulation parameters, this corresponds to a total bunch charge, $Q = eN$, of 0.37 nC, a remarkably successful prediction of the ZVP model. Perhaps counter-intuitively, it would appear the total charge scales inversely with the plasma density. Instead, charge can be increased either by increasing the laser pulse intensity or $L_z$. Indeed, provided the laser pulse intensity remains relativistic, the focal spot can be increased indefinitely and there is no limit to the mass-limited electron bunch total charge. For a realistic laser pulse with beam width $10 \lambda$ incident on a larger plasma block, Equation \ref{eq:zvp_N2} predicts a charge of 9.3 nC.

Caution is necessary when the standard definition of emittance is applied to a non-Gaussian beam profile, risking over- or underestimating the emittance. When applied to an ideal Gaussian distribution, the elliptical contour that defines the emittance contains 39.3 \% of the beam particles \cite{mcdonaldMethodsEmittanceMeasurement1989}, however, Figures \ref{fig:zvptypicalbunch} e) and f) contain 75.5 \% and 38.6 \% of the beam respectively. The large overestimate of the former is a consequence of the long low-density tail on the distribution to the bottom left of the plot.

Figure \ref{fig:zvptypicalbunchlines}a compares the electron bunch energy to an equivalent bunch produced by a circularly polarised laser pulse\footnote{A circularly polarised laser pulse will expel electrons from a mass-limited target in a corkscrew shape, the bunch is therefore only loosely equivalent.}. The mean electron bunch energy is over three times lower as there is no ZVP acceleration phase and there is no quasi-monochromatic structure \cite{baevaZeroVectorPotential2011}. Figure \ref{fig:zvptypicalbunchlines} b) plots the mean electron bunch energy in time at a point as the bunch passes, clearly demonstrating the attosecond duration of the electron bunch.
\begin{figure}
	\centering
	\includegraphics[width=1\linewidth]{figures/zvp/zvp_typical_bunch_lines}
	\caption[Energy spectra for mass-limited electron bunches formed via linearly and circularly polarised laser pulses.]{\textbf{Energy spectra for mass-limited electron bunches formed via linearly and circularly polarised laser pulses.}a) Energy spectra for mass-limited electron bunches formed via linearly and circularly polarised laser pulses. b) Mean energy density streaked in time through the centre of a mass-limited ZVP electron bunch demonstrating the attosecond duration.}
	\label{fig:zvptypicalbunchlines}
\end{figure}

\subsubsection{Applications: from electron bunches to attosecond light}
There is a plethora of applications for high charge, attosecond electron bunches, primarily for the purpose of resolving attosecond scale phenomena \cite{krauszAttosecondPhysics2009}. Already, femtosecond pump-attosecond probe experiments are underway \cite{calegariUltrafastElectronDynamics,takahashiNonlinearAttosecondMetrology2015} but the higher intensities and charge densities accessible in the laser-solid regime compared to in laser-gas interactions \cite{edwardsXRayEmissionEffectiveness2020, linIsolatedAttosecondElectron2020} would represent a step-change in the field by enabling atto-pump atto-probe experiments \cite{zhangGiantIsolatedAttosecond2020}. Potential applications include: electron microscopy and atomic diffraction to temporally resolve photoelectric processes such as Bragg diffraction \cite{morimotoDiffractionMicroscopyAttosecond2018}, ultra-fast electron radiography to probe the evolution of the formation of magnetic fields in dynamical systems \cite{schumakerUltrafastElectronRadiography2013} or for XFEL seeding \cite{cardenasSubcycleDynamicsRelativistic2019}: the record XFEL X-ray pulse duration stands at 280 attoseconds \cite{durisTunableIsolatedAttosecond2020}, substantially longer than the durations accessible using this technique. Electron bunches are also a promising alternative for radiotherapy due to their superior penetration depth in biotissues compared to X-rays \cite{glinecRadiotherapyLaserPlasma2006}.

% TODO Cite laser and plasma wakefield studies
Analogously to the \ac{HHG} process, rapid acceleration of an electron bunch generates a burst of radiation whose properties (brightness, coherency, duration, spectrum) are determined by the corresponding properties of the electron bunch (charge, emittance, duration and energy). Thus globally electron bunches are used as a diagnostic tool in synchrotrons and XFELs. Other acceleration mechanisms for X-ray generation include: bremmstrahlung radiation from firing the electron bunch at a secondary high-$Z$ target \cite{cordeFemtosecondRaysLaserplasma2013}, interaction with a counter-propagating laser pulse \cite{khrennikovTunableAllOpticalQuasimonochromatic2015,kulaginNonlinearReflectionHighamplitude2016} or injection into a laser or plasma wakefield accelerator, even with charge densities suitable for accession to the solid density plasma wakefield regime \cite{linIsolatedAttosecondElectron2020}. The mass-limited electron bunches produced via the ZVP mechanism have transverse emittances comparable in all planes to those conditioned in state-of-the-art nano-Coulomb electron bunch accelerators \cite{martinDiamondLightSource2020,pingProgressHEPSProject2020}. Such facilities typically produce electron bunches with geometric emittances of $\sim \unit{mm.rad}$ prior to damping ring injection \cite{christouPreInjectorLinacDiamond2004} and $\sim \unit{nm.rad}$ post-injection \cite{martinDiamondLightSource2020}. Thus, the mass-limited ZVP electron bunches are ideal candidates for the production of bright X-rays of unprecedentedly short duration and are suitable for study in the new attosecond regime with applications to physical, chemical and biological systems.

\subsubsection{Parameter scan of electron bunch mean energy}
It is important to confirm that mass-limited electron bunches follow the same scaling relations as has previously been confirmed for bulk ZVP electrons in both 1D \cite{baevaZeroVectorPotential2011} and 2D \cite{savinAttosecondscaleAbsorptionExtreme2017} PIC simulations. The mean mass-limited electron bunch kinetic energies were extracted from 120 2D PIC simulations and are plotted in Figure \ref{fig:zvpmeangammas}. 
\begin{figure}
	\centering
	\includegraphics[width=1\linewidth]{figures/zvp/zvp_mean_gammas}
	\caption[Mean mass-limited ZVP electron bunch normalised kinetic energies extracted from 2D PIC simulations.]{\textbf{Mean mass-limited ZVP electron bunch normalised kinetic energies extracted from 2D PIC simulations.} The bunch detailed in Figure \ref{fig:zvptypicalbunch} is highlighted.}
	\label{fig:zvpmeangammas}
\end{figure}
The dependence of the energy on both $a_0$ and $\bar{n}_\mathrm{e}$ is immediately apparent. Therefore, these electron bunches must be accelerated by a non-pondermotive mechanism. 

The parameter scan took plasma block densities ranging from the critical plasma density to well-beyond solid density for the aluminium target and with laser pulse peak intensities ranging from non-relativistic ($a_0 < 1$) through to the \ac{QED} plasma regime ($a_0 > 300$) up to a peak $a_0 = 5000$ to investigate the change in scaling observed by Savin \textit{et al} \cite{savinEnergyAbsorptionLaserQED2019} at the onset of QED effects. This study is also the first to extract specific bunch energies as opposed to total simulation box energy gain, representing a far more precise test of ZVP theory. Particle merging was turned on for macro-photons and macro-electrons for $a_0 \ge 1800$ to prevent overloading of the available supercomputer memory due to the vast number of particles produced in this SF-QED regime. 

A total of 856 data points for mass-limited electron bunch mean energies were extracted from the simulations. Bunches are arranged in trains with an average length of 4, however, isolated bunches are more useful for attosecond diagnostics. The number of bunches in a given train as a function of simulation peak intensity and plasma density is plotted in Figure \ref{fig:zvpbunchtrainlength}.
\begin{figure}
	\centering
	\includegraphics[width=0.7\linewidth]{figures/zvp/zvp_bunch_train_length}
	\caption[Mass-limited electron bunch train length as a function of laser intensity and plasma density.]{\textbf{Mass-limited electron bunch train length as a function of laser intensity and plasma density.} Lines of constant peak $S$ parameter are plotted for reference. Clearly, decreasing S reduces the bunch length.}
	\label{fig:zvpbunchtrainlength}
\end{figure}
It is evident that reducing the peak $S$ reduces the train length. Where $S <1$, the transition to relativistic self-induced transparency occurs in the rising edge of the laser pulse corresponding to a breakdown of ZVP and an early cutoff to the bunch train, suggesting one route to the more useful isolated attosecond electron bunches. Other routes to isolate electron bunches can be likely borrowed from CSE theory given the close relation between the two. The interest in isolated attosecond pulse of radiation has led to the proliferation of ideas, proposed techniques include the attosecond lighthouse technique using laser pulse wavefront rotation \cite{vincentiAttosecondLighthousesHow2012}, few-cycle laser pulses \cite{heisslerFewCycleDrivenRelativistically2012}, non-collinear laser pulse gating \cite{kennedyIsolatedUltrabrightAttosecond2022}, soon to be performed in experiment, and circular polarisation gating \cite{yeungDependenceLaserDrivenCoherent2014}.

Return now to Figure \ref{fig:zvpmeangammas}. The ZVP energy relations must be modified before comparison to the data since these electron bunches experience a further direct laser acceleration phase before reaching the measurement point. Thévenet \textit{et al} \cite{thevenetVacuumLaserAcceleration2016} suggested that attosecond electron bunches produced in reflection exhibit precisely the phase and energy properties required to `surf' the reflected laser pulse and experience vast acceleration gradients over the Rayleigh length of the laser pulse. This process is known as Vacuum Laser Acceleration and could provide a route to a fully optical scheme to create GeV nano-Coulomb electron bunches. Certainly, some of the bunch will have the necessary injection close to the laser propagation axis, however much of the bunch propagates at an angle to the laser and this must be accounted for. Interestingly, in Figure \ref{fig:experimentsetuphhgbunches2} modelling oblique incidence, more of the bunch appears to propagate along the axis. A parameter scan is necessary to find the optimal parameters.

Electrons are ejected from the plasma in phase with the subsequent laser pulse cycle peak and diverge outwards from that point. Near to the laser focus, the electrons experience a field
\begin{equation}\label{eq:Ey_ejection}
	\mathbf{E}(x,y,t) = E_0 e^{-(y-f_y)^2/w_\mathrm{L}^2}\cos(k_\mathrm{L}(x-f_x) - \omega_\mathrm{L}t) \mathbf{\hat{y}}= E_0 f(x,y,t)  \mathbf{\hat{y}}.
\end{equation}
The work done by this field is then
\begin{equation}\label{eq:deltaT_int}
	\Delta T = \int e \mathbf{E} \cdot \mathrm{d}\mathbf{x}.
\end{equation}
Note that for this process, the laser pulse electric field and electron bunch direction of travel will always be aligned no matter to which side of the plasma bulk the electrons are accelerated and therefore $\Delta T$ will always increase the energy of the electron bunch. The electron trajectories can be approximated as linear from the ejection point ($y_\mathrm{e},x_\mathrm{e}$) to the observation point ($y',x'$), correspondingly,
\begin{equation}
	x = y\frac{(x'-x_\mathrm{e})}{(y'-y_\mathrm{e})}, \ t = \sqrt{x^2 + y^2}/c.
\end{equation}
Equation \ref{eq:deltaT_int} can now be integrated along a given electron's path and therefore the gamma factor increases by
\begin{equation}
	\Delta\gamma = \int\frac{eE_y(y)\mathrm{d}y}{m_\mathrm{e}c^2} = a_0'(y)F,
\end{equation}
where $a_0'$ is the vector potential of the subsequent laser pulse cycle peak and 
\begin{equation}
	F = \frac{2\pi}{\lambda}\int f(y)\mathrm{d}y.
\end{equation}
The gamma factor gain from the ZVP acceleration phase is determined by the electric field at the corner of the plasma block,
\begin{equation}
	\Delta \gamma_\mathrm{ZVP} = \frac{(a_0 e^{-(2\lambda-y_\mathrm{f})^2/L_0^2})^2}{2\bar{n}_\mathrm{e}} = 0.31 \frac{a_0^2}{\bar{n}_\mathrm{e}}.
\end{equation}

Thus,
\begin{equation}\label{eq:gamma}
	\gamma = 1 + (0.31)\times \frac{a_0^2}{\bar{n}_\mathrm{e}} + F\times a_0'.
\end{equation}
The final term could be neglected or at least reduced somewhat once super-Gaussian spatial laser pulses become common in this intensity regime or by use of a suitable plasma separator \cite{miyauchiLaserElectronAcceleration2004}, as applied in \cite{andreevDoubleRelativisticElectron2013}. Both acceleration phases fail to meet the criteria of the Lawson-Woodward theorem. The ZVP phase is dependent on the existence of electrostatic forces, while the secondary phase occurs for a finite interaction region.

The data of Figure \ref{fig:zvpmeangammas} is taken from two measurement points, enabling two $F$ values to be investigated. Fits to Equation \ref{eq:gamma} were made for each data set using the Ordinary Least Squares regression model of the statsmodels Python module \cite{seaboldskipperandperktoldjosefStatsmodelsEconometricStatistical2010}, allowing the constants to vary. The non-linear ZVP energy can be applied to the linear model via construction of the necessary composite parameter. Both fits were successful with $r^2$-values of 0.81 and 0.84 and found the same ZVP pre-factor $F_\mathrm{ZVP} = 0.47 \pm 0.02$, slightly above that anticipated from the model. The model is sensitive to the laser pulse intensity, likely much of the bunch originated from closer to the laser pulse focus than the target corner explaining the higher value.

The first data set measurement point was (\qty{0.41}{\mu m}, \qty{0.25}{\mu m}) from the target edge. This gives $F = 0.31$ compared to $0.22 \pm 0.02$ predicted by the fit. The second, at (\qty{0.41}{\mu m}, \qty{0.31}{\mu m}) from the target edge giving $F = 0.39$ compared to $0.34 \pm 0.02$ from the fit.

This is the first demonstration of ZVP theory to calculate absolute values and not only the scaling relationship. Such order of magnitude calculation is essential for the comparison to other absorption models and the determination of dominant modes. It is certainly remarkable that such a simple theory for energy absorption has such predictive power in this highly non-linear and seemingly chaotic many-particle system. It is interesting that increasing laser intensity to such extremes will, at least for a short time, cause relativistic effects that simplify the dynamics before the total annihilation of a target.

The relative error between data and theory is plotted in Figure \ref{fig:zvp-logabserrortedit}. 
\begin{figure}
	\centering
	\includegraphics[width=1\linewidth]{figures/zvp/zvp_log_abs_error_T_edit.eps}
	\caption[The relative errors for each mean energy data point compared to Figure \ref{fig:zvpmeangammas}.]{\textbf{The relative errors for each mean energy data point compared to Figure \ref{fig:zvpmeangammas}.} The orange triangles indicate data points for which the model fails to predict the mean energy.}
	\label{fig:zvp-logabserrortedit}
\end{figure}
Those points marked by an orange triangle have associated errors of over an order of magnitude. Reassuringly, such points occur only after the onset of QED effects, known to impact the ZVP mechanism \cite{savinEnergyAbsorptionLaserQED2019}, and for $S<1$, that is, where the plasma becomes relativistically transparent to the laser pulse, a fundamentally different regime. The model is inconsistent in the non-relativistic domain as can be expected. To summarise, it would appear the ZVP model is valid for $10 \le a_0 \le 300$ and $S\ge1$.

\subsection{Energy absorption in the ZVP regime}\label{sec:zvp-energyabsorption}
% TODO replace normal incidence R expression with the full expression
As stated previously the laser-plasma coupling exists in a state of adiabaticity with the exception of the ZVP acceleration phase and hence Equation \ref{eq:zvp_U} describes the absorption of laser pulse energy. As two bunches are produced per laser pulse cycle, the rate of energy transfer is, therefore,
\begin{equation}\label{eq:zvp-rate}
	R = \frac{U\omega_\mathrm{L}}{\pi},
\end{equation}
for normal incidence. To observe the scaling for $U$ in 2D PIC simulations, peak instantaneous electron bunch energies escaping to the rear of the bulk were extracted from those PIC simulations with $S=1$. For constant $S$, 
\begin{equation}
	U \sim a^2_0.
\end{equation}
Energies are plotted in Figure \ref{fig:zvppeakgamma}.
\begin{figure}
	\centering
	\includegraphics[width=0.7\linewidth]{figures/zvp/zvp_peak_gamma}
	\caption[Peak instantaneous bulk electron bunch total energy escaping to the plasma bulk rear.]{\textbf{Peak instantaneous bulk electron bunch total energies after escaping to the plasma bulk rear.} Energies begin to deviate from the anticipated scaling at approximately $a_0 = 300$. Above $a_0 = 1000$, BW pair-produced electrons begin to dominate and the peak energies rise rapidly.}
	\label{fig:zvppeakgamma}
\end{figure}
Fitting the total energy within the range of validity established for the ZVP model finds
\begin{equation}
	U \sim a^{2.01\pm0.003}_0,
\end{equation}
reproducing with great success the anticipated scaling within the ZVP regime. It is satisfying that the number of electrons and bunch mean energies both follow their anticipated ZVP scalings. It was not possible to reproduce the constants of Equation \ref{eq:zvp_U} as the neutralising return current in the plasma bulk generates an electrostatic field on the rear side of the plasma block, decelerating bulk electron bunches as they escape the plasma. It should be possible to calculate the deceleration by considering the number of electrons expelled by the plasma. It is, however, clear from the simulations that at least some electrons in the escaping bunch are trapped by this rear-side potential well and thus reduce its ability to slow electrons.

% TODO get a citation for the resistive return current
While Equation \ref{eq:zvp_U} describes energy absorption into hot electron bunches, the coupling of such hot collisionless electrons to the bulk plasma, given the lack of collisionality, must be indirect. There are two key mechanisms \cite{sherlockIndepthPlasmawaveHeating2014}. Firstly, via a cooler resistive return current of electrons that neutralises the current of the injected hot electrons that escape the potential well of the front surface. Since all hot electrons travel at approximately speed $c$, the magnitude of the return current depends not on the total energy absorbed but instead on the total number of electrons injected, as given by Equation \ref{eq:zvp-Ne}. The current depends thus linearly on laser spot area and the electric field magnitude and, unexpectedly, not on the plasma density\footnote{Note that for a sufficiently thin target, the return current induces an electrostatic field on the back surface of the target which can then reflect hot electron bunches and decelerate them to the point of a return to collisionality. This is a reality for the PIC simulations explored in this thesis, however, since realistic targets are much thicker this shall be neglected.}. Secondly, dissipation of energy occurs via the formation of large amplitude bulk plasma waves induced in the wake of the hot electron bunches. Sherlock \textit{et al} \cite{sherlockIndepthPlasmawaveHeating2014} calculate the magnitude of the induced wakefield to be
\begin{equation}
	E_\mathrm{W} = \frac{eN_\mathrm{e}c}{\omega_\mathrm{p}\epsilon_0} = \sigma \sqrt{\frac{m_\mathrm{e}\epsilon_0}{n_\mathrm{e}}}E_\mathrm{L},
\end{equation}
where here the bunch velocity has been set to $c$, bulk electrons will be accelerated by $E_\mathrm{W}$ and their kinetic energy converted to heat via collisions. Interestingly, this reproduces the mid-temperature electron scaling with density that was observed by Chrisman \textit{et al} \cite{chrismanIntensityScalingHot2008} in their study of hot electron energy coupling in cone-guided fast ignition of inertial fusion targets. This is a different possible explanation to their self-declared `hand waving argument'. Excluding this study, such formulations for heat transfer to the plasma bulk within the ZVP regime remain untested in simulations.

Note also that as the laser pulse intensity rises, the fraction of energy absorbed by the ion species via hole boring increases. Savin \cite{savinModellingLaserPlasmaInteractions2019} determined for $S=1/2$, $a_0 = 100$, that this would be almost 20\%.

\subsection{Unpacking the QED effects of Figure \ref{fig:zvppeakgamma}}
In Savin's acclaimed paper \cite{savinEnergyAbsorptionLaserQED2019}, they determined theoretically and demonstrated in simulation that at $a_0 = 300$, $n_\mathrm{e} = 50 n_\mathrm{c}$, there is a transition from standard ZVP scalings to an enhanced QED scaling associated with \ac{BW} electrons increasing the pseudocapacitor plate charge. Explicitly,
\begin{equation}
	T \sim \frac{a^5_0}{\bar{n}_\mathrm{e}}.
\end{equation}
At first glance of Figure \ref{fig:zvppeakgamma}, Savin's results are inconsistent with this parameter scan. Perhaps the measurement method can explain this via the well-known effect of radiation trapping from \ac{RR} \cite{jiRadiationReactionTrappingElectrons2014}, also observed in these PIC simulations. After acceleration across the pseudocapacitor, the electron bunch encounters the subsequent laser peak. If the electron bunch gamma factor and laser intensity are both large enough, electrons radiate a significant fraction of their energy and are thus stopped in their tracks. Unable now escape the potential well at the plasma surface they are trapped and are not observed to escape the plasma until the laser pulse intensity reduces. Such an effect would not impact Savin's scalings but would of course inhibit the observation of the scaling for electrons to the rear of the plasma block. Note that should this be the case and the collisionless electron bunches remain within the plasma bulk, Savin's ZVP QED model truly applies directly to energy absorption by the plasma bulk.

There is another interpretation. Returning now to Figure \ref{fig:zvppeakgamma}, there are two interesting aspects. Firstly, there is a sudden jump in total energy at $a_0 \approx$ 300. This cannot be explained by ZVP theory nor QED theory since the jump is observed in the absence of Smilei's QED modules. Secondly, there is an even stronger jump in total electron bunch energy above $a_0 = 1000$. Decomposing the total energy into bulk electrons and those produced via the BW process, this is clearly a QED effect. Energy in BW produced electrons scales at a staggering
\begin{equation}
	U_\mathrm{BW} \sim a^{11.3\pm 0.8}_0,
\end{equation}
while the energy of bulk plasma electrons decreases. Perhaps this is a signal of Savin's QED ZVP electron bunches only at a higher energy due to the substantially greater plasma density of these simulations. The reduction in bulk electron energy can be attributed to an oversaturation of the front surface with BW electrons.

Combining equations \ref{eq:intro-photonchi} and \ref{eq:zvp_U}, and assuming the electron radiates all its energy to the photon, for the ZVP mechanism at the point of emission one finds
\begin{equation}
	\chi_\gamma =\frac{ \sqrt{2} |\mathbf{E}|}{E_\mathrm{S}}\frac{a^2_0}{\bar{n}_\mathrm{e}}.
\end{equation}
The probability of \ac{BW} pair production begins to rapidly increase around $\chi =1$, therefore, the transition to QED will occur at
% TODO these potentially wrong, need to double check if the factor of two is needed in expressions for gamma factor.
\begin{equation}
	a_0 \approx \left(\frac{ E_\mathrm{S}}{\sqrt{2}}\frac{e\bar{n}_\mathrm{e}}{m_\mathrm{e}c\omega_\mathrm{L}}\right)^{1/3} = \left(\frac{a_\mathrm{s}\bar{n}_\mathrm{e}}{\sqrt{2}}\right)^{\frac{1}{3}},
\end{equation}
where $a_\mathrm{S} = \num{7.73e5}$ is the normalised vector potential associated with the Schwinger Field. Repeating the calculation instead as a function of $S$, one finds instead
\begin{equation}
	a_0 \approx \left(\frac{a_\mathrm{s}S}{\sqrt{2}}\right)^{\frac{1}{2}}.
\end{equation}
This corresponds to a transition to rapid pair production with Savin's parameters of $a_0 \approx 301$ versus $a_0 \approx 739$, consistent with both studies. This is interesting, ZVP with Savin parameters predicts a significantly lower transition to QED than could be expected from the Wilks scaling for $\mathbf{J}\times\mathbf{B}$ heating and could be tested at ELI-np.

Following Savin's QED ZVP theory, one can predict
\begin{equation}
	U_\mathrm{QED} \sim \frac{a^7_0}{S},
\end{equation}
therefore much work remains to understand the scaling of Figure \ref{fig:zvppeakgamma} and to unify these results, starting with the application of the methods of this analysis to the parameter space explored by Savin. Undoubtedly, the advent of next-generation exa-watt scale lasers will be exceedingly interesting if such scalings in bunch energy can be maintained.

Linear Breit-Wheeler can safely be neglected in these simulations. There is simply not enough energy in the system. For two interacting photons of energy $E_1$, $E_2$, by consideration of four-momenta, the threshold condition for pair production is 
\begin{equation}
	E_1E_2 \ge (m_\mathrm{e}c^2)^2.
\end{equation}
For a near-infrared laser photon of energy $\hbar \omega_\mathrm{L}$, the interacting photon must have an energy in excess of 200 GeV. Despite the extreme acceleration gradients considered in this thesis, the ZVP mechanism predicts photons of energies no greater than 10 GeV and thus linear Breit-Wheeler is suppressed.

He \textit{et al} \cite{heSinglelaserSchemeObservation2021} identified an alternative mechanism for linear \ac{BW} using solid density targets, where forwards- and back-scattered high energy radiation reaction-produced photons interact within a hollowed-out plasma channel. In the geometry of interest in this thesis, while photons are produced in both directions, their production is localised to the plasma surface and at no points do their paths cross preventing the occurrence of this mechanism.

% TODO: Explore transition regions, and do MORE simulations to fill in the gaps.

% TODO To do: convergence of electron bunch energies with nppc and sim resolution.

\subsection{Errors}
A set of simulations was undertaken to explore the stability of some of the PIC simulation assumptions. The effect of including collisions, increasing the temperature to resolve the Debye length and particle merging were explored. The results are presented in Figure \ref{fig:zvperrorsources}. 
% TODO: Make this plot fraction errors instead
\begin{figure}
	\centering
	\includegraphics[width=1\linewidth]{figures/zvp/zvp_error_sources}
	\caption[An exploration of ZVP 2D PIC simulation stability.]{\textbf{An exploration of ZVP 2D PIC simulation stability and assumptions. }a) Electron bunch mean energies extracted from simulations with $a_0$, $\bar{n}_\mathrm{e} =100$. Here, the $x$-axis is the normalised vector potential of the laser pulse cycle that made them, this is why there appears to be results stacked. b) Electron bunch mean energies extracted from simulations with $a_0$, $\bar{n}_\mathrm{e} =1800$ both with and without particle merging.}
	\label{fig:zvperrorsources}
\end{figure}
In general, including collisions had little to no effect on the results and changing the Debye length had a small but not systematic impact on the results. Note that while numerical heating is not directly dependent on laser intensity, increases to $a_0$ would increase the electron bunch density. Thus, resolving the Debye length of the ultra-high density (orders of magnitude greater than solid density) electron bunches produced by the higher laser pulse intensities considered here is doubtful, although the high energy of such particles may mitigate this issue. Particle merging was essential for simulations with $a_0 > 1800$ due to the proliferation of high energy photons and electron-positron pairs. At least at $a_0 = 1800$, it would appear particle merging had negligible impact, however, it is likely the error would increase with $a_0$.

% TODO Also perhaps put in the plot of a laser of constant intensity and variation between bunches.

% TODO Also look again at fitting just to the region where the expression is valid.



\section{Conclusions}\label{sec:zvp-conclusion}
% TODO: 3D3V mention somewhere
The Zero Vector Potential mechanism describes the post-ponderomotive rapid absorption of ultra-relativistic laser energy by a solid density overdense, collisionless and fully ionised plasma on the timescales of ion immobility. The defining characteristics of the mechanism have been identified in 3D PIC simulations including the observation of a zero of the vector potential propagating at speed $\approx 1.4c$ through a high-density ZVP electron bunch at the front surface of the plasma early in the ablative journey of the electron bunch. Simulations have suggested that from currently operational 10 PW short pulse laser facilities and foam targets, the ZVP mechanism can produce a train of attosecond duration, nano-Coulomb electron bunches, each with a transverse emittance of a few $\unit{nm.rad}$. Such charge and quality are comparable to state-of-the-art electron bunch accelerators but on paradigm-shifting timescales. Such timescales being those on which atomic processes occur, these electron bunches could be manipulated to literally `shed light' onto fundamental biological and chemical processes. Via a massive 2D PIC parameter scan the energies of such mass-limited electron bunches have been compared to those predicted by the ZVP model, identifying a range of validity for the model, specifically $a_0>10$, $S>1$. These simulations were also used to confirm the energy absorption scaling in 2D up to and into the QED regime, posing new ideas and questions for the theoretical understanding of ZVP QED. Some discussion was given on the modes of energy absorption and on simulations exploring various of the assumptions in the model.

Chapter \ref{ch:4-gemini} details the upcoming GEMINI PW experiment to observe ZVP electron bunches for the first time. On the theme of experimentation, the following Chapter switches gears from absorption to reflection for the discussion of recent results on the ORION laser facility.