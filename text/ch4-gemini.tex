\chapter{\label{ch:4-gemini}Attosecond electron bunches and X-ray pulses on GEMINI PW} 

\minitoc

\section{Overview}

We successfully measured the intensity of the attosecond duration X-ray harmonics on the ORION SP1 and SP2 beamlines. However, the uncertainties remained large, the geometry sub-optimal and the pulse train exceedingly far from an isolated attosecond pulse. Indeed none of the proposed mechanisms for isolation (polarisation gating, lighthouse technique, etc.) could have any hope of success. The high shot rate, few femtosecond, highly customisable GEMINI PW facility at CLF can address all these points while also providing conditions suitable for the production of attosecond ZVP electron bunches. Naturally accompanied by its own repertoire of technical challenges. The work of the prior chapters of this thesis formed the body of a successful proposal for five weeks of beam time at the facility with shot dates planned for Summer 2024. This chapter outlines the extensive planning process initiated over a year before the experiment itself.

There are three primary experimental goals. First, to repeat the ORION experiment on GEMINI PW with optimised geometry and thus resolve and measure the absolute intensity of X-ray harmonics, detailed in Section X. Second, to simultaneously observe ZVP electron bunches in transmission and the HHG signal they generate in reflection, detailed in Section X. Finally, to apply the X-ray harmonic beam in a proof of principle Laue diffraction setting described in Section X. Section X addresses the subject of laser contrast, a much thornier issue for the few femtosecond GEMINI beamline compared to ORION and covers the parallel experiment proposed by collaborators at Queen's University Belfast to provide some illuniation to the problem at hand. Note that no petawatt class few femtosecond laser facility has observed harmonics at oblique incidence. This section will also detail the strategy of contingency should the contrast prove to be too poor for the production of harmonics.

% TODO replace HHG with SHHG

CAD drawings overview where to put you?

The experimental geometry is relatively simple. Post \ac{DPM} contrast enhancement, the GEMINI PW South (S) beam is focused by an $f/2$ parabola onto the solid density target at \qty{45}{\degree} \ac{AOI} and p-polarisation, directing the specularly reflected X-ray beam to the West side of the chamber. XUV and X-ray spectrometers aligned to the target in the specular direction will observe the reflected harmonic beam.

\section{X-ray harmonics on GEMINI PW}
Naturally, the first action is to repeat the ORION experiment at GEMINI PW. For this purpose, the Oxford Engineering Department is designing and building a replica OHREX spectrometer to perform the measurement with OHREX crystals borrowed from AWE. The lower energy KAP (100) OHREX crystal listed in Table \ref{tab:dispersion} is more suited to GEMINI PW due to the lower energy of the beamline compared to ORION. Regardless, the quartz crystals will also be fielded. 

A 1D PIC simulation of the GEMINI PW beamline for the planned geometry predicts harmonics will be resolved for the KAP energy range.
[INSERT PIC SIMS HERE].


The technical complexity of the OHREX spectrometer is entirely attached to the spherically bent OHREX crystals, thus reducing the challenges of alignment \cite{beiersdorferLineshapeSpectroscopyVery2016}. It is designed to be relatively insensitive to the distances between source and crystal (2.4 m) and from crystal to image plane (0.524 m). Indeed, previous experiments deemed it unnecessary to adjust the OHREX bellows to access the best focus. Variation in the angle of incidence from \qty{38.7}{\degree} shifts the spectral image away from the nominal energy range. 
% TODO am I doing this? The variation is relatively sensitive, for the quartz ($10\bar{1}1$) crystal, a change in energy of \qty{8}{\%} is obtained by a change in angle of \qty{50}{\mu rad} \cite{macdonaldAbsoluteThroughputCalibration2021}. Nevertheless, t
The alignment of the crystal planes to the crystal surface enables the relative convenience of optical alignment. Initial shots will use IP to check for any light leakage and to confirm alignment. Switching to the Raptor Photonics Eagle XV in vacuum X-ray CCD camera (EA4240XV-BN-CL) \cite{EagleXVVacuum} to utilise the high shot rate on GEMINI PW. The CCD camera must be liquid cooled, thus for the prevention of damaging condensation forming on the camera window a manual gate valve is required to isolate the OHREX when pumping the main chamber (waiting for the camera to reach acceptable temperatures while under vacuum would be prohibitively long). With $2048 \times 2048$ active pixels of size \qty{13.5}{\mu m} $\times$ \qty{13.5}{\mu m}, not all of the 4 cm crystal image can be captured by the camera.


A parameter scan of \ac{XHHG} as a function of $S$ can be performed by varying the laser intensity, via beam apodisation, and the target material. The majority of shots will use target wheels of the relatively high damage threshold fused silica with the plan to advance to a kapton tape. The fused silica targets are chemically etched in one corner to aide alignment.
% TODO cite the below
Since HHG is suppressed for circularly polarised light, bremmstrahlung emission and/or other X-ray production mechanisms can somewhat be ruled out by comparing the \ac{XHHG} signal observed for p- and circularly- polarised light. However, this is not fully the case at \qty{45}{\degree} \ac{AOI} and plasma heating is reduced, reducing the temperature of the emitting plasma.

Filtering is necessary to reject the high intensity low order harmonics, however, for photon energies in the 600 eV range, the standard beryllium filter transmission is too low. Instead 400 nm of aluminium flash-coated onto 1 micron of mylar will be used for the KAP crystal, corresponding to a transmission of \qty{16}{\%} at 600 eV.

It is essential to check there is suitable resolution in both the spectrometer and the CCD camera for harmonic observation. For the highest energy crystal, nominal photon energy of 2.405 keV, 96 harmonics sit in the range accessed by the crystal, corresponding to a fractional energy $\Delta E/E =$ \num{6.45e-4} between harmonics. At  $\Delta E/E =$ \num{1e-4}, the OHREX crystal resolution is sufficient, however it may not resolve the harmonics' spectral shape. Those 96 harmonics are spread over the 4 cm crystal image. At a resolution of \qty{13.5}{\mu m}, the CCD camera has 31 pixels between harmonics and the harmonic width is about a pixel.

% TODO KAPTON sims
sims:
Running new SiO2 HHG sim, hopefully this will show that we have a better efficiency than the previous simulation. Then pop that in here.

A previous experiment measured the GEMINI DPM setup produced a reduction in peak pulse intensity of \qty{50}{\%}. Applying ROM and hole boring theory to a fused silica target and accounting for filtering and assuming no merging of harmonics, one can anticipate signals at the OHREX crystal position of \qty{19.6}{\mu J.sr^{-1}} per harmonic at 0.6 keV for the KAP crystal and \qty{1.89}{\mu J.sr^{-1}} per harmonic at 2.405 keV for the quartz ($10\bar{1}1$) crystal. Orienting the OHREX for s-polarisation to maximise the signal at the image plane, this corresponds to an average of 3.4 photons per pixel and 0.20 photons per pixel respectively. The quantum efficiency of the detector is close to \qty{100}{\%}. These numbers are no smaller than the ORION experiment while the background will be lower due to the order of magnitude lower energy on target. Thus, there should be adequate statistics to observe these signals.

\section{First observation of ZVP electron bunches}



ZVP bunches 
3 omega imaging line
CTR theory


\section{Application of HHG to Laue diffraction}
Laue diffraction



\section{Contrast}

Contrast DPM, GEMINI issues, probe beam, potential solutions, contingency (currently in Resolve X-ray HHG section)

Kahaly prepulse


In this initial stage we will use a UV spectrometer. As of 2019, no experiment to confirm HB theory had been performed on a PW system \cite{vincentiAchievingExtremeLight2019}. These UV spectrometer measurements could be very interesting both to confirm the HB theory in the PW regime and to infer the intensity increase at PM focus, for this we would ideally resolve the first 50 orders.
\section{Conclusion}



% TODO add optimal HHG conditions somewhere

% TODO compare as pulse brightness to gas as source