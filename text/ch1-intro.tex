\chapter{\label{ch:1-intro}Introduction} 

\minitoc

%{A plan}


%sections to include
% Start with the laser and our entrance into the multi-petawatt regime with no signs of stopping - ref 40 in Alex Savin's thesis - it looks like that is saying huge growth in laser power check that and maybe write in the thesis but don't include the figure. Also, include a description of CPA and OPCPA
%intense lasers and absorption mechanisms?? yes for sure but don't do all those just below ZVP
%Frames of reference - lab, sim, ablating front surface of plasma

% What is the story? Based on the first section by Alex after the abstract
%Plasma is ubiquitous in our known universe and plasma provides us with huge opportunities as a tool to improve our lives
%Simulataneously our ability to understand physics has been aided by an explosion in computing power (Peter HEDP paper)
%In this thesis we discuss some of the opportunities that relativistic laser plasma physics offers us with solid density targets - note that note about solids v gases at this point. 
%Perhaps even before the debye length, define what we mean by the temperature of the plasma.

%An unused statement about ion immobility
%Assume for now that the ion-electron mass ratio is infinite, that is to say, the ions are approximately immobile for the timescales under consideration, generally true for a fair few relativistic laser pulse cycles (In later sections the mobility of plasma ions will prove very important but for now this is ignored.).
% I should also discuss ionisation and laser surface plasma interaction./

\section{High-energy-density physics}
We sit on the verge of the exascale revolution, in both computing resources \cite{kotheExascaleComputingUnited2019} and laser power. With rapid adoption of the pioneering advancements of Chirped Pulse Amplification (CPA) by Strickland and Mourou \cite{stricklandCompressionAmplifiedChirped1985} and of Optical Parametric Chirped Pulse Amplification \cite{rossOpticalParametricChirped2000}, multi-petawatt facilities have been commissioned and built across the globe and plans for sub-exawatt facilities are now underway. Figure \ref{fig:intro_lasersystems} details the parameter space accessible by current and proposed facilities \cite{stricklandCompressionAmplifiedChirped1985, fourmauxPedestalCleaningHigh2011, wang085PWLaser2017, tiscarenoOhioStateUniversity2020, sung42PW202017, rossGenerationTerawattPulses2000, yangMultiterawattLaserSystem2002, witte2006source, stoecklHighEnergyPetawattProject2006, lozhkarevCompact56Petawatt2007, herrmannGenerationSubthreecycle162009, andriukaitis90GWPeak2011, zhaoGeneration120GW2013, hoppsOverviewLaserSystems2013, hoppsOverviewLaserSystems2013, thire10MJ5cycle2015, yinHighefficiencyOpticalParametric2016, galesExtremeLightInfrastructure2018, OMEGAFacility, renCompactDoublepassRaman2008, lanciaExperimentalEvidenceShort2010, lanciaSignaturesSelfSimilarRegime2016, marquesJouleLevelHighEfficiencyEnergy2019, thebergeTunableUltrashortLaser2006, trushinSub10fsSupercontinuumRadiation2007, ohIntenseTerahertzGeneration2013, horioGenerationSub17Fs2014, takahashiGeneration10uJCoherent2002, goulielmakisSingleCycleNonlinearOptics2008, skantzakisCoherentContinuumExtreme2009, ferrariHighenergyIsolatedAttosecond2010, takahashiAttosecondNonlinearOptics2013, popmintchevUltravioletSurpriseEfficient2015, nomuraAttosecondPhaseLocking2009, bierbachGeneration10Relativistic2012, heisslerMultimJHarmonicEmission2015, yeungExperimentalObservationAttosecond2017, jahnIntenseIsolatedAttosecond2019, rusEfficientHighbrightnessSoftxray1997, sebbanFullCharacterizationHighgain2000, rusMultimillijouleHighlyCoherent2002, zeitounHighintensityHighlyCoherent2004, wangHighBrightnessInjectionSeededSoftXRayLaser2006, rohringerAtomicInnershellXray2012, ayvazyanFirstOperationFreeelectron2006, ackermannOperationFreeelectronLaser2007, emmaFirstLasingOperation2010, FlashFreeElectronLaser, FlashFreeElectronLaser, FELIXLaboratoryOverview, FELIXLaboratoryOverview, FELIXLaboratoryOverview, kneipObservationSynchrotronRadiation2008, kneipBrightSpatiallyCoherent2010, juEnhancementXraysGenerated2012, chenBrightBetatronXray2013, wangQuasimonoenergeticLaserplasmaAcceleration2013, coleLaserwakefieldAcceleratorsHard2015, wenzQuantitativeXrayPhasecontrast2015, taphuocAllopticalComptonGammaray2012, chenMeVEnergyRaysInverse2013, tsaiCompactTunableCompton2015, polyanskiyPicosecondPulseAmplification2011, haberbergerFifteenTerawattPicosecond2010, glowniaAmplification193nmFemtosecond1992, kandoEnhancementPhotonNumber2009, obenschainHighenergyKryptonFluoride2015, trinesSimulationsEfficientRaman2011, trinesSimulationsEfficientRaman2011, kawanakaConceptualDesignSubexawatt2016, tajimaMarriage20keVSuperconducting2018}, clearly, there is no indication of wavering interest or support for the field of \ac{HED} physics. 
\begin{figure}
	\centering
	\includegraphics[width=\linewidth]{figures/intro/laser_systems}
	\caption[Laser systems across the globe, both commissioned, planned and theorised. theorised.]{A by no means exhaustive plot of high power light producing facilities from across the globe, both commissioned, planned and theorised \cite{stricklandCompressionAmplifiedChirped1985, fourmauxPedestalCleaningHigh2011, wang085PWLaser2017, tiscarenoOhioStateUniversity2020, sung42PW202017, rossGenerationTerawattPulses2000, yangMultiterawattLaserSystem2002, witte2006source, stoecklHighEnergyPetawattProject2006, lozhkarevCompact56Petawatt2007, herrmannGenerationSubthreecycle162009, andriukaitis90GWPeak2011, zhaoGeneration120GW2013, hoppsOverviewLaserSystems2013, hoppsOverviewLaserSystems2013, thire10MJ5cycle2015, yinHighefficiencyOpticalParametric2016, galesExtremeLightInfrastructure2018, OMEGAFacility, renCompactDoublepassRaman2008, lanciaExperimentalEvidenceShort2010, lanciaSignaturesSelfSimilarRegime2016, marquesJouleLevelHighEfficiencyEnergy2019, thebergeTunableUltrashortLaser2006, trushinSub10fsSupercontinuumRadiation2007, ohIntenseTerahertzGeneration2013, horioGenerationSub17Fs2014, takahashiGeneration10uJCoherent2002, goulielmakisSingleCycleNonlinearOptics2008, skantzakisCoherentContinuumExtreme2009, ferrariHighenergyIsolatedAttosecond2010, takahashiAttosecondNonlinearOptics2013, popmintchevUltravioletSurpriseEfficient2015, nomuraAttosecondPhaseLocking2009, bierbachGeneration10Relativistic2012, heisslerMultimJHarmonicEmission2015, yeungExperimentalObservationAttosecond2017, jahnIntenseIsolatedAttosecond2019, rusEfficientHighbrightnessSoftxray1997, sebbanFullCharacterizationHighgain2000, rusMultimillijouleHighlyCoherent2002, zeitounHighintensityHighlyCoherent2004, wangHighBrightnessInjectionSeededSoftXRayLaser2006, rohringerAtomicInnershellXray2012, ayvazyanFirstOperationFreeelectron2006, ackermannOperationFreeelectronLaser2007, emmaFirstLasingOperation2010, FlashFreeElectronLaser, FlashFreeElectronLaser, FELIXLaboratoryOverview, FELIXLaboratoryOverview, FELIXLaboratoryOverview, kneipObservationSynchrotronRadiation2008, kneipBrightSpatiallyCoherent2010, juEnhancementXraysGenerated2012, chenBrightBetatronXray2013, wangQuasimonoenergeticLaserplasmaAcceleration2013, coleLaserwakefieldAcceleratorsHard2015, wenzQuantitativeXrayPhasecontrast2015, taphuocAllopticalComptonGammaray2012, chenMeVEnergyRaysInverse2013, tsaiCompactTunableCompton2015, polyanskiyPicosecondPulseAmplification2011, haberbergerFifteenTerawattPicosecond2010, glowniaAmplification193nmFemtosecond1992, kandoEnhancementPhotonNumber2009, obenschainHighenergyKryptonFluoride2015, trinesSimulationsEfficientRaman2011, trinesSimulationsEfficientRaman2011, kawanakaConceptualDesignSubexawatt2016, tajimaMarriage20keVSuperconducting2018, NetElectricityConsumption, umstadterRelativisticLaserPlasma2003, dollarEnhancedLaserAbsorption2017} to provide an overview of the parameter space currently and soon to be accessible. The experimental result obtained at the ORION laser facility \cite{hoppsOverviewLaserSystems2013} is included, and will be discussed in detail in chapter \ref{ch:3-orion}. The dashed orange line is from the relevant theoretical model. The red dashed line marks the parameter space attainable via the methods discussed in this thesis when coupled with a sub-etawatt laser pulse. Such a beamline would provide intensities in the water window many orders of magnitude beyond that currently available from state-of-the-art facilities. }
	\label{fig:intro_lasersystems}
\end{figure}

\ac{HED} physics is the laboratory study of the behaviour of matter with a pressure above \qty{10}{GPa}, approximately one million atmospheres, and containing free electrons not confined to a solid state \cite{drakeFocusHighEnergy2014}, typically in the plasma state of matter. \ac{HED} conditions are found for a vast range of densities and temperatures (from zero to a million million Kelvin) and operating in both the quantum and relativistic realms. The applications are equally diverse, including but not limited to inertial confinement fusion, particle acceleration for scientific or medical purposes and light sources as diagnostic tools. Ubiquitous in the natural universe, beyond our solar system, all that can be observed in the sky is radiation emitted from \ac{HED} plasmas \cite{chenIntroductionPlasmaPhysics2016}.

Independently, the field of attosecond physics has flourished \cite{krauszAttosecondPhysics2009}. By irradiating gas targets with mildly ionising laser pulse intensities, pulses of attosecond radiation, composed of high-order harmonics of the laser, have been generated. Recognised by the 2023 Nobel Prize for Physics, the research of Agostini, Krausz and Huillier have utilised this \ac{HHG} to unveil the inner workings of atoms and molecules, tracking fundamental electronic dynamics \cite{ElectronilluminatingLaserPulses2023}. 

At the intersection of HED and attosecond physics, there is another course for the generation of harmonics with the potential to produce brighter and shorter pulses than those of the laser-gas interaction. This thesis concerns itself with that interaction, that of a high-power short-pulse laser incident on a flat solid target. Through this \ac{SHHG}, state-of-the-art 10 PW laser facilities such as ELI-np \cite{galesExtremeLightInfrastructure2018} can produce electron bunches and light pulses of exceptional charge and brightness, and of attosecond duration, thus uncovering new avenues for attosecond resolution diagnostics and for probing vacuum non-linearities. 

Seemingly counter-intuitively, as the laser power increases, via relativistic effects and for particular initial conditions, greater coherency in electron dynamics can be observed and signals are amplified. Before delving into this fascinating phenomenon, the remainder of this chapter provides some of the relevant background information. Chapter \ref{ch:2-zvp} introduces the \ac{ZVP} model of attosecond absorption to describe this laser-solid interaction. Chapter \ref{ch:3-orion} focuses instead on the reflected radiation and its high harmonic content, specifically with regard to the measurement of X-ray harmonics on the March 2023 ORION experiment. This chapter accessed a moderately different regime to that of \ac{ZVP}. Chapter \ref{ch:4-gemini} brings together both absorption and reflection in the ZVP regime, detailing the upcoming GEMINI PW experiment. Finally, Chapter \ref{ch:5-summary} summarises the body of work covered by this thesis and discusses potential future work.

\section{Electromagnetism fundamentals}
The spatiotemporal propagation of electric $\mathbf{E}(t,\mathbf{x})$ and magnetic $\mathbf{B}(t,\mathbf{x})$ fields must satisfy Maxwell's equations \cite{chenIntroductionPlasmaPhysics2016}
\begin{subequations}
	\label{eq:intro-maxwell}
	\begin{align}
		\nabla \cdot \mathbf{B} &= 0, \label{eq:intro-B0} \\
		\nabla \cdot \mathbf{E} &= \frac{\rho}{\epsilon_0},\label{eq:intro-E0} \\
		\nabla \times \mathbf{B} &= \mu_0 \mathbf{J} + \mu_0 \epsilon_0 \partial_t \mathbf{E},\label{eq:intro-B1} \\
		\nabla \times \mathbf{E} &=-\partial_t \mathbf{B}. \label{eq:intro-E1}
	\end{align}
\end{subequations}
Here, $\epsilon_0 = \qty{8.85e-12}{F.m^{-1}}$ and $\mu_0 = \qty{1.26e-6}{N.A^{-2}}$ are the vacuum permittivity and permeability respectively and $\rho(t,\mathbf{x})$ and $\mathbf{J}(t,\mathbf{x})$ the total charge and current densities of the charged particles present in the system. 

A particle with charge $q$ and velocity $\mathbf{v}$ in the presence of electromagnetic fields experiences the Lorentz force,
\begin{equation}\label{eq:intro-Lorentz_force}
	\mathbf{F}_\mathrm{L} = q(\mathbf{E} + \mathbf{v} \times \mathbf{B}).
\end{equation}

The electromagnetic fields can be obtained from the scalar, $\phi$, and vector, $\mathbf{A}$, potentials as  \cite{steaneRelativityMadeRelatively2012}
\begin{equation}
	\mathbf{E} = -\nabla \phi - \partial_t \mathbf{A},
\end{equation}
\begin{equation}
	\mathbf{B} = \nabla \times \mathbf{A}.
\end{equation}

\subsubsection{The Vlasov-Maxwell system of equations}\label{sec:ch1-vlasov}
A collisionless and fully ionised plasma is fully described in the kinetic description by the Vlasov-Maxwell system of equations \cite{derouillatSmileiCollaborativeOpensource2018}. Each plasma species, $s$, of particles with mass $m_s$ and charge $q_s$ is represented by its distribution function $f_s(t,\mathbf{x},\mathbf{p})$ at time $t$, position $\mathbf{x}$ and momentum $\mathbf{p} = m_s \gamma \mathbf{v}$. The distribution satisfies the Vlasov equation, that is,
\begin{equation}\label{eq:intro-vlasov}
	(\partial_t + \frac{\mathbf{p}}{m_s\gamma} \cdot \nabla + \mathbf{F}_\mathrm{L} \cdot \nabla_\mathbf{p})f_s = 0,
\end{equation}
where $\mathbf{F}_\mathrm{L}$ is the Lorentz force given in equation \ref{eq:intro-Lorentz_force}. The electric $\mathbf{E}(t,\mathbf{x})$ and magnetic $\mathbf{B}(t,\mathbf{x})$ fields that generate the force must satisfy Maxwell's Equations (Equations \ref{eq:intro-maxwell}).

This self-consistent system of equations describes the dynamics of plasma particles within electromagnetic fields. The particles modify the fields via their charge and current densities,
\begin{equation}\label{eq:intro-rho}
	\rho(t,\mathbf{x}) = \sum_s q_s \int \mathrm{d}^3pf_s(t,\mathbf{x},\mathbf{p}),
\end{equation}
and 
\begin{equation}\label{eq:intro-J}
	\mathbf{J}(t,\mathbf{x}) = \sum_s q_s \int d^3\mathrm{p}\mathbf{v}f_s(t,\mathbf{x},\mathbf{p}),
\end{equation}
respectively.

\section{\label{sec:plasma_def}The definition of a classical plasma}
As outlined in F. Chen's definitive textbook `Introduction to Plasma Physics and Controlled Fusion' \cite{chenIntroductionPlasmaPhysics2016}, a plasma must fulfil three criteria, namely,
\begin{enumerate}
	\item Ionisation: a plasma must consist of both charged and neutral particles. Of course, this alone cannot define a plasma, any gas will contain some degree of ionisation. Note, with reference to figure \ref{fig:intro_lasersystems} \cite{umstadterRelativisticLaserPlasma2003}, that upon incidence a modern high-power laser system will instantaneously fully ionise a target.
	\item Quasineutrality: while locally there can be (often extreme) electromagnetic forces and charge concentrations at work, over the length scales of the plasma, such forces are screened out and the plasma bulk remains net neutral in charge.
	\item Collective behaviour: unlike in a gas, where collisions dominate the dynamics, the particles in a plasma generate electromagnetic fields that interact at a distance. Thus a particle's motion depends not only on its immediate vicinity but on the surrounding plasma conditions. Indeed, it is often the so-called \textit{collisionless} plasmas, where collisions can be safely neglected, that are of most interest, as is the focus of this thesis.
\end{enumerate}

These conditions can be quantitatively defined by the Debye length, the plasma parameter and the plasma frequency as laid out in the following sections.

\subsection{\label{sec:debye_length}The Debye length}
The Debye length describes the extent to which a plasma can shield electromagnetic fields within and so remain quasineutral. Consider an infinitely extended plasma with a test charge placed at some point. What then is the scalar potential $\phi(\mathbf{x})$ around it? If the plasma had no kinetic energy, the charged particles would arrange themself immediately adjacent to the test charge and once this equilibrium state was reached there would be no electromagnetic fields present. More realistically, the plasma will have some temperature, likely a very large temperature, and so some particles will be able to escape the potential of the test charge and thus leak electromagnetic fields into the plasma bulk. Poisson's equation (equation \ref{eq:intro-E0} in the static case) reads
\begin{equation}\label{eq:poisson}
	\epsilon_0\nabla^2\phi = -e(Zn_\mathrm{i} - n_\mathrm{e}),
\end{equation}
where $\epsilon_0 = \qty{8.854e-12}{F.m^{-1}}$ is the permittivity of free space, $e = \qty{1.602e-19}{C}$ is the charge of an electron, $Z$ is the plasma ion charge in units of $e$ and $n_\mathrm{i}$ and $n_\mathrm{e}$ are the number densities of plasma ions and electrons respectively.

Since the electrons are significantly more mobile than the ions due to their lower mass, it is in general the electrons and not the ions that respond to the test charge and the ions can be assumed to provide a constant background of positive charge density.
If the number density of electrons follows a Boltzmann temperature distribution in the presence of a potential energy $-e\phi$, then
\begin{equation}\label{eq:nj_boltzmann}
	n_\mathrm{e}= n_{\mathrm{e},0}e^{e\phi/KT_\mathrm{e}},
\end{equation}
where $n_{\mathrm{e},0}$ is the electron number density far from the test charge, $n_\mathrm{i} = n_{\mathrm{e},0}/Z$, $K = \qty{1.38e-23}{J.K^{-1}}$ is the Boltzmann constant and $T_\mathrm{e}$ is the electron temperature in Kelvin. Note that in plasmas it is very common for different species to have differing temperatures depending on the mechanism for energy absorption and the timescales for collisions compared to the timescale of the study.

Substituting equation \ref{eq:nj_boltzmann} into equation \ref{eq:poisson} and Taylor expanding the exponential term in the limit that the plasma is weakly coupled ($e\phi << KT_\mathrm{e}$), 
\begin{equation}\label{eq:poisson_debye2}
	\nabla^2\phi = \frac{\phi}{\lambda_\mathrm{D}^2},
\end{equation}
where
\begin{equation}\label{eq:debye}
	\lambda_\mathrm{D} \equiv \sqrt{\frac{\epsilon_0KT_\mathrm{e}}{n_\mathrm{e}e^2}},
\end{equation}
is the \textit{Debye length} and describes the thickness of the charge sheath surrounding the test charge. For quasineutrality to hold for the plasma bulk, its spatial dimensions, $L$, must extend beyond a few Debye lengths, \textit{i.e.}
\begin{equation}
	L \gg \lambda_D.
\end{equation}

\subsection{\label{sec:plasma_parameter}The plasma parameter}
In order for the derivation of section \ref{sec:debye_length} to be statistically valid, there must be a large number of charged particles within the shielding sheath. The number of particles within the \textit{Debye sphere} is
\begin{equation}\label{eq:plasma_parameter}
	N_\mathrm{D} = \frac{4}{3}\pi\lambda_\mathrm{D}^3n,
\end{equation}
where, $N_\mathrm{D}$ is the \textit{plasma parameter}. Note that, as discussed already, in most cases it is most suitable to choose the number density $n$ to be the number density of electrons, $n_\mathrm{e}$. To ensure the plasma is suitably ionised (criterion 1) and that the plasma engages in collective behaviour (criterion 3),
\begin{equation}\label{eq:plasma_parameter_condition}
	N_\mathrm{D} \ggg 1.
\end{equation}
Quite the reverse is necessary for the creation of a quantum plasma.

\subsection{\label{sec:plasma_frequency}Collisionality and the plasma frequency}
Collective behaviour depends not only on the ability for large numbers of particles to interact via electromagnetic forces but also that these forces dominate over collisions in describing particle trajectories. Taking $\omega$ as the typical frequency of plasma oscillations and $\tau$ as the average time between collisions, for a plasma (as opposed to a gas)
\begin{equation}\label{eq:plasma_frequency_condition}
	\omega\tau > 1
\end{equation}
is required. It now remains to determine what is the typical frequency of collisions in a given plasma. While the types of plasma waves and their associated frequencies of oscillation are multitudinous, the characteristic frequency, the \textit{plasma frequency}, $\omega_\mathrm{p}$, naturally arises from the most straightforward. It describes the response of electrons to charge imbalances within an infinite uniform plasma at rest in the absence of magnetic fields or temperature fluctuations. As in Section \ref{sec:debye_length}, the ions provide a constant background of positive charge.

Consider a semi-infinite plasma existing for $x>0$, with electron density $n_\mathrm{e}$ and ion density $n_\mathrm{e}/Z$ of charge state $Z$\footnote{This description has direct relevance to the Zero Vector Potential mechanism which will be made clear in Chapter \ref{ch:2-zvp}.}. Suppose the electron fluid is displaced by some perfectly isotropic force into the plasma bulk a distance $(\Delta x) \hat{\mathbf{x}}$ as in Figure \ref{fig:introplasmafrequency}. 
\begin{figure}
	\centering
	\includegraphics[width=0.7\linewidth]{figures/intro/intro_plasma_frequency}
	\caption[A diagram to illustrate the derivation of plasma frequency.]{\textbf{A diagram to illustrate the derivation of plasma frequency.} The electrons of a semi-infinite plasma are displaced inwards by some external force leaving in their wake an uncompensated space charge of `immobile' positive ions. By constructing a Gaussian surface along the dashed line, the electric field associated with the positive space charge can be calculated from Gauss' Law.}
	\label{fig:introplasmafrequency}
\end{figure}
The total charge of displaced electrons within a surface area of $\sigma$ is 
\begin{equation}\label{eq:intro_Q}
	Q = -en_\mathrm{e}\sigma\Delta x.
\end{equation}
Applying the integral form of Gauss' law (from equation \ref{eq:intro-E0}) to the surface described in Figure \ref{fig:introplasmafrequency}, the uncompensated charge leads to 
\begin{equation}\label{eq:intro_E}
	-\sigma E\hat{\mathbf{x}}= \frac{Q}{\epsilon_0}\hat{\mathbf{x}} = -\frac{en_\mathrm{e}\sigma\Delta x}{\epsilon_0}\hat{\mathbf{x}}
\end{equation}
at the plasma surface. By the Lorentz force, Equation \ref{eq:intro-Lorentz_force}, the displaced electrons will experience a restoring force, $-eE\hat{\mathbf{x}}$, perpendicular to the surface due to the electron-ion charge imbalance. The equation of motion for electrons on that surface is therefore
\begin{equation}\label{eq:intro_sho}
	m_e\frac{\mathrm{d}^2\Delta x}{\mathrm{d}t^2} = -eE = -\frac{e^2n_\mathrm{e}}{\epsilon_0}\Delta x.
\end{equation}
Equation \ref{eq:intro_sho} clearly describes a simple harmonic oscillator with a characteristic frequency given by the plasma frequency,
\begin{equation}
	\omega_\mathrm{p} = \sqrt{\frac{e^2n_\mathrm{e}}{m_\mathrm{e} \epsilon_0}}.
\end{equation}
Interestingly, the collisionality condition is always satisfied for a classical Maxwellian plasma.

\section{The Lawson-Woodward theorem}\label{sec:intro-lawson_woodward}
The Lawson-Woodward theorem states that there can be no net electron energy gain from laser fields \cite{esareyPhysicsLaserdrivenPlasmabased2009}, quite at odds with one of the primary aspects of this thesis, that is, the acceleration of electrons. There are, however, several conditions that are frequently met, namely,
\begin{enumerate}
	\item The interaction region is infinite.
	\item The interaction occurs in a vacuum.
	\item The electron is ultra-relativistic ($v\approx c$) along the acceleration gradient.
	\item No electro- or magnetostatic fields are present.
	\item Non-linear effects are neglected.
\end{enumerate}
Several of these apply to the various accelerations of electrons considered here. It is this final condition that is most damning to the application of the theorem. Throughout this thesis, the ultra-relativistic laser pulses under consideration ensure non-linear effects cannot be neglected. It is indeed such non-linearities that are of most interest.

\section{Laser-solid density plasma linear interaction}\label{sec:intro-lasersolidplasma_linear}
Consider small transverse electromagnetic waves propagating through a plasma. Linearising equation \ref{eq:intro-Lorentz_force} for a single plasma electron by assuming only small field variations and thus small velocity variation,
\begin{equation}\label{eq:intro-EOM_linear}
	m_\mathrm{e}  \dot{\mathbf{v}}_\mathrm{e} = -e\mathbf{E}.
\end{equation}
Combining the time derivative of Equation \ref{eq:intro-B1} and the curl of Equation \ref{eq:intro-E1},
\begin{equation}
	\nabla\times\nabla \times \mathbf{E}= - \mu_0 \dot{\mathbf{J} }- \mu_0\epsilon_0 \ddot{\mathbf{E}}.
\end{equation}
Considering only fast oscillations, such that ions are effectively immobile,
\begin{equation}
	\mathbf{J} = - n_\mathrm{e} e\mathbf{v}_\mathrm{e}
\end{equation}
and using the identity $\nabla\times\nabla \times \mathbf{E} = \nabla(\nabla\cdot \mathbf{E}) - \nabla^2\mathbf{E}$,
\begin{equation}
	\nabla(\nabla\cdot \mathbf{E}) - \nabla^2\mathbf{E} = -\frac{\mu_0 n_\mathrm{e} e^2}{m_\mathrm{e}} \mathbf{E}  - \mu_0 \epsilon_0 \ddot{\mathbf{E}}.
\end{equation} 
Assuming plane wave solutions of the form
\begin{equation}
	\mathbf{E} = \mathbf{E_0}e^{i(\mathbf{k}\cdot \mathbf{x}-\omega t)},
\end{equation}
where $\mathbf{k}$ is the wave-vector and $\omega$ the frequency of oscillations and noting the waves are transverse $\mathbf{k}\cdot \mathbf{E} = 0$,
\begin{equation}
	|\mathbf{k}|^2 \mathbf{E} = -\frac{\mu_0 n_\mathrm{e} e^2}{m_\mathrm{e}}\mathbf{E} + \mu_0\epsilon_0\omega^2\mathbf{E}
\end{equation}
and hence the dispersion relation for electromagnetic waves propagating in a plasma is
\begin{equation}\label{eq:intro-EM_dispersion}
	\omega^2  = c^2|\mathbf{k}|^2  + \omega_\mathrm{p}^2.
\end{equation}
Equation \ref{eq:intro-EM_dispersion} exhibits a \textit{cutoff} dependent on the plasma density via $\omega_\mathrm{p}$. The \textit{critical density}, $n_\mathrm{c}$, is defined as the density above which a laser pulse of frequency $\omega_\mathrm{L}$ cannot propagate through a plasma. This occurs for $\omega_\mathrm{L} = \omega_\mathrm{p}$, thus,
\begin{equation}
	n_\mathrm{c} = \frac{m_\mathrm{e}\epsilon_0 \omega_\mathrm{L}^2}{e^2}.
\end{equation}
As the plane wave has a spatial dependence $\sim \exp{(\mathbf{k}\cdot\mathbf{x})}$, if $n_\mathrm{e} > n_\mathrm{c}$, $\mathbf{k}$ is imaginary and the wave no longer propagates through the plasma and instead exponentially attenuates over a skin depth,
\begin{equation}
	\delta = \frac{1}{|\mathbf{k}|} = \frac{c}{\sqrt{\omega_\mathrm{p}^2 - \omega_\mathrm{L}^2}}
\end{equation}
and is reflected. For typical high-power lasers with wavelengths in the visible or near-infrared, fully ionised solids tend to have densities well above the critical density and thus produce plasma mirrors.

\section{Relativity}
Modern high-power lasers operate in the domain of relativistic mechanics and in general interactions are highly non-linear. It is useful to introduce some of the basic principles of relativity. There has been growing interest in the curvature of spacetime from relativistic lasers \cite{atongaGravitationalWavesHighpower2023}, however, this effect remains undetectable at present. Thus, throughout this thesis, the inner product of 4-tensors is defined using the Minkowski Metric \cite{steaneRelativityMadeRelatively2012}. 

Many useful quantities can be arranged into contravariant four-vectors that undergo a Lorentz transformation for a change of frame of reference \cite{steaneRelativityMadeRelatively2012}, specifically, 
\begin{equation}
	\mathbf{A}'^\mu =\Lambda_\mu^\nu \mathbf{A}^\mu,
\end{equation}
where $\Lambda^\mu_\nu$ is the appropriate Lorentz transformation, and primed symbols typically denote boosted frames of reference. Without loss of generality, the coordinate system can be defined such that the boosted frame travels along the $\mathbf{x}$-axis with respect to the initial frame. Thus, the Lorentz transformation is defined as
\begin{equation}\label{eq:zvp_lorentz}
	\Lambda_\mu^\nu = \begin{pmatrix}
		\gamma & -\beta\gamma & 0 & 0\\
		-\beta\gamma & \gamma & 0 & 0\\
		0 & 0& 1 & 0\\
		0 & 0 & 0 & 1
	\end{pmatrix}.
\end{equation}
Generally, a \textit{beta factor} is a normalised speed or velocity of an object,
\begin{equation}
	\beta = \frac{v}{c}, 
\end{equation}
here it refers to the frame velocity and its associated Lorentz or \textit{gamma factor} is
\begin{equation}
	\gamma = \frac{1}{\sqrt{1-\beta^2}}.
\end{equation}
Four-vectors relevant to this thesis are listed in Table \ref{tab:intro-four-vectors}.
%Many useful quantities can be arranged into contravariant four-vectors that undergo a Lorentz transformation for a change of frame of reference, those relevant to this thesis are: the four-displacement
%\begin{equation}
%	\mathbf{X}^\mu = (ct, \mathbf{x}),
%\end{equation}
%the four-potential
%\begin{equation}
%\mathbf{A}^\mu = \left(\frac{\phi}{c}, \mathbf{A}\right),
%\end{equation}
%the four-current density
%\begin{equation}
%	\mathbf{J}^\mu = (c\rho,\mathbf{J}),
%\end{equation}
%the four-wave vector
%\begin{equation}
%	\mathbf{K}^\mu = \left(\frac{\omega}{c}, \mathbf{k}\right)
%\end{equation}
%and the four-momentum of a particle
%\begin{equation}
%	\mathbf{P}^\mu = \left(\frac{U}{c}, \mathbf{p}\right),
%\end{equation}
%where $U = \gamma m c^2$ is the particle energy and $\mathbf{p} = \gamma m\mathbf{v}$ its three-momentum.
\begin{table}
	\begin{center}
		\begin{tabular}{llll}
			\hline \hline
			Symbol & Name & Components & Invariant \\
			\hline
			$\mathbf{X}^\mu$& 4-displacement & $(ct, \mathbf{x})$ & $-c^2\tau^2$  \\
			$\mathbf{A}^\mu$&4-potential  & $(\phi/c, \mathbf{A})$  &  \\
			$\mathbf{J}^\mu$& 4-current density & $(c\rho,\mathbf{J})$ & $-c^2\rho_0^2$ \\
			$\mathbf{K}^\mu$& 4-wave vector &$(\omega/c, \mathbf{K})$  &  \\
			$\mathbf{P}^\mu$& 4-momentum &  $(U/c, \mathbf{p})$& $-m^2c^2$ \\
			\hline \hline
		\end{tabular}
		\caption{\label{tab:intro-four-vectors} \textbf{Four-vectors of relevance to this thesis.} New parameters are the proper time, $\tau$, the proper charge density, $\rho_0$, energy, $U = \gamma m c^2$, three-momentum, $p = \gamma m\mathbf{v}$.}
	\end{center}
\end{table}
Transformations of electromagnetic fields under reference frame boosts are given in Appendix \ref{sec:app_lorentzEM}. Maxwell's equations are Lorentz covariant.
% TODO ~ ensure all mentions of energy use U for bulk and T for individual

Focusing now on the 4-potential $\mathbf{A}^\mu$ and choosing the Lorenz gauge,
\begin{equation}
	\partial_\mu \mathbf{A}^\mu = \nabla \cdot \mathbf{A} + \frac{1}{c^2}\partial_t \phi = 0,
\end{equation}
then Maxwell's equations can be written
\begin{equation}\label{eq:intro-wave_equation}
	\partial_\nu \partial^\nu \mathbf{A}^\mu = -\frac{1}{c^2\epsilon_0}\mathbf{J}^\mu.
\end{equation}
Equation \ref{eq:intro-wave_equation} can be solved to yield 
\begin{equation}
	\mathbf{A}(\mathbf{x},t) = \frac{\mu_0}{4\pi} \int \frac{\mathbf{J}(\mathbf{x'},t_\mathrm{a})}{|\mathbf{x}-\mathbf{x}'|} \mathrm{d}^3\mathbf{x}',
\end{equation}
where $t_\mathrm{a} = t- |\mathbf{x}-\mathbf{x}'|/c$ is the advanced time. Hence, electromagnetic fields radiated from charged particles in motion can be calculated\footnote{Or even, as is performed in the full technical derivation relevant for the work of Chapter \ref{ch:3-orion}, electromagnetic fields can be described in terms of the future charge motions they will incite \cite{baevaHighHarmonicGeneration2008}.}.

% TODO ~ I think this is a good spot for the relativistic Larmor frequency

\subsection{Ultra-relativistic similarity theory}
Consider a relativistically intense laser pulse normally incident on a collisionless plasma as in Figure \ref{fig:introplasmafrequency}. Again neglect ion motion. The electron distribution is fully described by the Vlasov equation (Equation \ref{eq:intro-vlasov}) with the self-consistent electric and magnetic fields satisfying Maxwell's equations (Equations \ref{eq:intro-maxwell}). Suppose the incident laser pulse has an initial vector potential 
\begin{equation}
	\mathbf{A}(t=0) = \mathbf{a}((y^2+z^2)/R^2,x/c\tau)\cos k_\mathrm{L}x.
\end{equation}
This envelope form for the potential, $\mathbf{a}((y^2+z^2)/R^2,x/c\tau)$, is sensible provided $k_\mathrm{L}R \gg 1$ and $\omega_\mathrm{L}\tau \gg 1$, where $R$ is the focal spot radius and $\tau$ the pulse duration. For fixed laser envelope, the laser-plasma dynamics depend on just four dimensionless variables:  the normalised focal spot size, $k_\mathrm{L} R$, the normalised pulse duration, $\omega_\mathrm{L}\tau$, the normalised laser vector potential amplitude
\begin{equation}
	a = \mathrm{max}\left|\frac{e \mathbf{A} }{m_\mathrm{e} c^2 }\right|,
\end{equation}
in terms of the peak laser electric field amplitude $\mathbf{E}_\mathrm{L}$,
\begin{equation}
	a_0 = \frac{e|\mathbf{E}_\mathrm{L}|}{m_\mathrm{e} c\omega_\mathrm{L}},
\end{equation}
and the normalised plasma density
\begin{equation}
	\bar{n}_\mathrm{e} = \frac{n_\mathrm{e}}{n_\mathrm{c}}.
\end{equation}
By normalising the system of equations and combining these last two expressions into the \textit{relativistic similarity parameter},
\begin{equation}
	S = \frac{\bar{n}_\mathrm{e}}{a_0},
\end{equation}
it is possible to show that in the ultra-relativistic limit ($a_0 \gg 1$), the dynamics of the system are similar for constant $S$ \cite{gordienkoScalingsUltrarelativisticLaser2005} with plasma electrons following similar trajectories where 
\begin{equation}\label{eq:intro-p_similarity}
	\mathbf{p} \sim a_0.
\end{equation}
There is also a more physical meaning to the $S$ parameter. Consider again Section \ref{sec:intro-lasersolidplasma_linear} on the propagation of linear electromagnetic waves through a plasma but now for the case of an ultra-relativistic laser pulse. For an electron rotating in an electromagnetic field,
\begin{equation}
	\mathbf{F}_\perp = \gamma m_e \mathbf{a}_\perp,
\end{equation}
where $ \mathbf{a}_\perp$ is the acceleration perpendicular to the motion and thus the response of the electrons is reduced by a factor of $\gamma$. While some find the \textit{relativistic mass} correction to be somewhat unhelpful nomenclature for the phenomenon \cite{steaneRelativityMadeRelatively2012}, it has nevertheless become commonplace within the literature of relativistic plasma physics \cite{umstadterRelativisticLaserPlasma2003}. Turning the handle, one finds that the relativistic plasma frequency is
\begin{equation}
	\omega_\mathrm{p}^\mathrm{rel} = \sqrt{\frac{e^2n_\mathrm{e}}{\gamma m_\mathrm{e} \epsilon_0}}.
\end{equation}
Using equation \ref{eq:intro-p_similarity}, and taking $v \approx c$, then $\gamma \approx a_0$ and the normalised relativistic cutoff density is simply $S$. Thus, the ultra-relativistic similarity parameter is simply a measure of the overdensity of a plasma once relativistic corrections have been applied, \textit{i.e.} for $S>1$, a laser pulse will be reflected, however for $\bar{n}_\mathrm{e} > 1$ and $S <1$, one enters the regime of relativistically self-induced transparency \cite{ereminRelativisticSelfInducedTransparency2010}. It is now possible to define the parameter space of interest in this thesis: relativistic laser-plasma surface interactions occur for $a_0 \gg 1$ and $S > 1$.

\subsection{Relativistic lasers and plasmas}
The descriptor \textit{relativistic} is applied liberally in this thesis. When applied to electromagnetic fields or laser pulses it refers to 
\begin{equation}
	a_0 \ge 1.
\end{equation}
When applied to particles, their Lorentz factors are
\begin{equation}
	\gamma = \frac{1}{\sqrt{1-\beta^2}} \ge 2,
\end{equation}
corresponding to a speed, $u \ge 0.87 c$. \textit{Ultra-relativistic} implies these quantities are much larger than the conditions provided. A relativistic laser pulse will accelerate electrons to relativistic velocities in a fraction of a laser pulse cycle. Consider an electron in the presence of a uniform electric field of magnitude $a_0 = 100$, an intensity accessible by current state-of-the-art laser facilities. The work done on that particle by the field is 
\begin{equation}
	T = (\gamma -1)m_\mathrm{e}c^2 = \int \mathbf{E}\cdot \mathrm{d} \mathbf{x},
\end{equation}
The field will accelerate an electron to relativistic velocities in a distance less than 1 \% of a corresponding laser pulse wavelength. It is reasonable therefore to assume the interaction is instantaneously relativistic.

Upon entrance to the relativistic regime, the motion of an electron fundamentally changes. Consider Equation \ref{eq:intro-Lorentz_force}. For non-relativistic laser pulses, the magnetic field component can be neglected and the electron simply oscillates along the electric field vector direction. Once electron velocities approach $c$, this approximation is no longer valid. Electrons are rotated in the magnetic field and accelerated along the laser propagation direction. In a plasma, this enables inwards compression of the surface, leading to laser-induced hole boring \cite{wilksAbsorptionUltraIntenseLaser1992}.

\subsection{Conservation of generalised transverse momentum}\label{sec:intro_conservation-generalised-mometum}
Consider a holonomic system of $N$ relativistic particles under the influence of electromagnetic forces. A particle $j$ with charge $e_j$ and mass $m_j$ experiences a scalar potential,
\begin{equation}
	V_{j} = e_j(\phi - \mathbf{A} \cdot \mathbf{v}_{j})
\end{equation}
and hence the system is described by the Lagrangian \cite{goldsteinClassicalMechanics2013}
\begin{equation}
	\mathcal{L} = \sum^N_{j=1}\left( - m_jc^2\sqrt{1-\beta^2_\mathrm{j}} - e_j(\phi - \mathbf{A} \cdot \mathbf{v}_j) \right),
\end{equation}
The generalised momentum corresponding to coordinate $x_j$ is
\begin{equation}
	p_{j,x} = \frac{\partial \mathcal{L} }{\partial \dot{x}_j} = \frac{m_j\dot{x}_j }{\gamma_j}+ e_jA_x,
\end{equation}
describing both the linear mechanical momentum and the momentum of the electromagnetic field. Via Noether's theorem, if $L$ is independent of $x_j$, \textit{i.e.} spatially homogeneous along $x$ for particle $j$, then 
\begin{equation}\label{eq:intro-transverse_momentum_differential_equation}
	\dot{p}_{j,x} = 0
\end{equation}
since
\begin{equation}
	\frac{\mathrm{d}}{\mathrm{d}t}\left(\frac{\partial \mathcal{L} }{\partial \dot{x}_j}\right) = \frac{\partial L}{\partial x_j}.
\end{equation}

Taking the Lorenz gauge, consider a linearly polarised Gaussian laser pulse, with an axis of polarisation along $x$ incident on a solid target at rest. Then $A_x$ is approximately constant along $x$ near the beam centre\footnote{Constant relative to the scale of typical electron trajectories in such an interaction.}. Integrating Equation \ref{eq:intro-transverse_momentum_differential_equation}, fully constraining the gauge by setting the potential to zero initially, and noting that there is no linear momentum at the target, the generalised transverse momentum conservation equation for an electron in the laser field is
\begin{equation}\label{eq:intro-transverse_momentum_conservation_no_initial_momentum}
	p_\mathrm{T} = eA,
\end{equation}
where $p_\mathrm{T}$ is the electron momentum along the polarisation axis of the laser pulse and $A$ is the laser pulse 3-vector potential amplitude. As a sanity check, this expression complies with the ultra-relativistic similarity result of Equation \ref{eq:intro-p_similarity}.

Note that this is only valid provided the electron does not radiate along the direction of polarisation as discussed by Sokolov \textit{et al} \cite{sokolovDynamicsEmittingElectrons2009}. The implications of \textit{Radiation Reaction }are discussed in the following section.

\section{QED effects}
Next-generation laser facilities will enable the testing of decades-old theoretical predictions of \ac{SF-QED}. Already Fedeli \textit{et al} have shown in simulations that current PW-class laser facilities can access this regime using an all-optical set-up based on laser-solid surface interactions \cite{fedeliProbingStrongfieldQED2020}. The Schwinger Limit $E_\mathrm{S} = \qty{1.32e18}{V.m^{-1}}$ is the field intensity at which the vacuum non-linearity can produce real particles. If by some means an electron can be directed towards a plane electromagnetic wave, by consideration of the Lorentz transformations of electromagnetic fields (equations \ref{eq:app-maxwell_transformation}), it is possible that provided the electron is sufficiently relativistic, in its own reference frame it will `see' electromagnetic fields intense enough to access such vacuum non-linearities. The first two frontiers of SF-QED that will be accessed are Radiation Reaction and multi-photon Breit-Wheeler electron pair production. Brief introductions to these phenomena are now presented.

\subsection{High-energy photon emission and radiation reaction}
When a charged particle undergoes an acceleration, it emits electromagnetic radiation. If the electromagnetic field is sufficiently strong, \textit{i.e.} approaching the Schwinger Limit in the rest frame of the particle, then a non-negligible fraction of the particle momentum can be transferred to the emitted high energy photon, substantially impacting the dynamics of the accelerated particle. This back reaction is known as \ac{RR}.

Smilei (detailed in the following section) implements the process of high-energy photon emission as Inverse Compton Scattering on the basis of several assumptions \cite{nielQuantumClassicalModeling2018}, namely,
\begin{enumerate}
	\item Radiating particles are ultra-relativistic and therefore radiation is emitted in the direction of travel of the particle.
	\item The field varies slowly over the timescale of photon emission, this is the \textit{locally-constant field approximation} and requires ultra-relativistic field strengths;
	\item but they are small with respect to the Schwinger Limit, specifically requiring the invariants $\sqrt{c^2\mathbf{B^2} + \mathbf{E}^2}$ and $\sqrt{c\mathbf{B}\cdot\mathbf{E}} < E_\mathrm{S}$.
	\item Real particles radiate independently of their neighbours, this requires the emitted wavelength to be shorter than the typical inter-particle spacing.
\end{enumerate}
Provided such conditions hold, the rate of photon emission depends on two invariants \cite{ritusQuantumEffectsInteraction1985}, the electron quantum parameter
\begin{equation}
	\chi = \frac{\gamma}{E_\mathrm{S}}\sqrt{(\mathbf{E} + \mathbf{v}\times\mathbf{B})^2- (\mathbf{v}\cdot\mathbf{E})^2/c^2},
\end{equation}
where $\mathbf{v}$ is the electron velocity and the emitted photon quantum parameter
\begin{equation}\label{eq:intro-photonchi}
	\chi_\gamma = \frac{\gamma_\gamma}{E_\mathrm{S}}\sqrt{(\mathbf{E}_\perp + \mathbf{c}\times\mathbf{B})^2- (\mathbf{c}\cdot\mathbf{E})^2/c^2},
\end{equation}
where $\gamma_\gamma$ is the normalised photon energy $=\hbar \omega_\gamma /m_\mathrm{e}c^2$. The exact relationship is complex and in the fully quantum domain ($\chi > 1$), is it not practical to solve the integrations required for all particles. Instead, values are extracted from precalculated tables and supplied to a Monte Carlo algorithm. 

\subsection{Multi-photon Breit-Wheeler pair production}
Multi-photon Breit-Wheeler pair production, also known as non-linear Breit-Wheeler is the decay of a high energy photon, typically produced via \ac{RR}, into an electron-positron pair in the presence of a strong electromagnetic field, explicitly,
\begin{equation}
	\gamma + n\omega \to e^- + e^+.
\end{equation}
The strength of the effect is dependent on the Lorentz invariant photon quantum parameter, Equation \ref{eq:intro-photonchi}. In a constant electric field, the rate of pair production increases rapidly up to $\chi_\gamma \approx 10$ at which point it saturates and slowly reduces.

\section{Simulating the interaction}
Modelling laser-plasma interactions is a notoriously challenging endeavour. Due to the complexity of the many-bodied systems involved (a fully ionised centimetre cubed block of plastic contains on the order of $10^23$ particles) and the stochasticity of particle motion, it is frequently impossible to construct models \textit{ab initio}. Instead, hydrodynamic simulation codes such as HYADES \cite{larsenHYADESPlasmaHydrodynamics1994} and FLASH \cite{fryxellFLASHAdaptiveMesh2000} and Particle-In-Cell simulation codes such as Smilei \cite{derouillatSmileiCollaborativeOpensource2018}, Osiris \cite{fonsecaOSIRISThreeDimensionalFully2002} and EPOCH \cite{bennett2017users} are used to construct phenomenological models and to direct experimentation. 

\subsection{Supercomputing resources}\label{sec:intro-archer}
Modern \ac{HPC} systems are poised to enter the exascale regime ($> 10^{18}$ Floating Point Operations Per Second). With limited improvements in microprocessor technologies, such power is achieved through massive parallelisation across processing units. Able to study the dynamics of billions of macroparticles, PIC codes test the limits of modern supercomputing architectures. ARCHER2, the UK's national supercomputer, came online in November 2021, with it delivering over ten times the resources of its predecessor (ARCHER) \cite{ARCHER2}. An HPE Cray EX supercomputing system with a peak performance estimated at \qty{28}{Pflops.s^{-1}} across 5860 nodes each with dual AMD EPYCTM 7742 64-core processor for a total of 750,080 cores, ARCHER2 was able to supply the resources required to run the costly PIC simulations for this research. The substantially cheaper HYADES simulations were performed on the Rutherford Appleton Laboratory's SCARF \ac{HPC} cluster \cite{SCARFOverview}.

\subsection{Particle-In-Cell codes}

\subsubsection{Discretisation of the Vlasov-Maxwell equations}
Finding numerical solutions to the Vlasov-Maxwell equations is no straightforward. While codes exist that are capable, such as Valis \cite{sircombeVALISSplitconservativeScheme2009}, the requirement of high resolution in both position and momentum is exceedingly costly and the use of such codes is limited with respect to their size, duration and number of spatial dimensions. A more tractable approach is to discretise the distribution function into $N_s$ \textit{quasi-particles}\footnote{Originally introduced by Langdon and Birdsall as \textit{clouds} \cite{langdonTheoryPlasmaSimulation1970}.}. These are often referred to as \textit{macro-particles} in practice and typically represent a large number of real particles, such that
\begin{equation}
	f_s(t,\mathbf{x},\mathbf{p}) = \sum^{N_s}_{p=1} w_p S(\mathbf{x}-\mathbf{x}_p(t))\delta (\mathbf{p}-\mathbf{p}_p(t)),
\end{equation}
where $w_p$ is the quasi-particle's weight, $\mathbf{x}_p$ and $\mathbf{p}_p$ are its position and momentum respectively, $\delta$ is the Dirac-delta distribution and $S(\mathbf{x})$ the shape-function chosen to represent the spatial extent of the quasi-particle. The Vlasov equation is then integrated along the continuous trajectories of quasi-particles while Maxwell's equations are solved on a discrete spatial grid of \textit{cells}. Such a code is aptly named a \textit{Particle-In-Cell} (PIC) code. A schematic of the standard PIC code algorithm is presented in Figure \ref{fig:intropiccycle-01}. 
\begin{figure}
	\centering
	\includegraphics[width=\linewidth]{figures/intro/intro_PIC_cycle}
	\caption[A schematic of the PIC code loop and the algorithms performed.]{A schematic of the PIC code loop and the algorithms performed from time step, $ts$, from $n$ to $n+1$.}
	\label{fig:intropiccycle-01}
\end{figure}
After particle and field initialisation, fields are interpolated at particle positions. The well-established momentum-conserving \textit{Boris pusher} algorithm computes the new macro-particle velocities and positions \cite{borisRelativisticPlasmaSimulationoptimization1970}. Particles are advanced in time using a \textit{leap-frog }scheme, where positions are defined at integer, $n$, time steps and momenta at half-integer, $n + 1/2$. The charge conserving Esirkepov algorithm \cite{esirkepovExactChargeConservation2001} projects the new charge and current densities onto the grid to then solve Maxwell's equations using the Finite-Difference-Time-Domain approach \cite{tafloveComputationalElectromagneticsFiniteDifference2005}. To ensure space and time centring of the electromagnetic field derivatives in Maxwell's equations, electric and magnetic fields are discretised on the staggered \textit{Yee grid} as represented in Figure \ref{fig:introyeegrid}. 
\begin{figure}
	\centering
	\includegraphics[width=0.7\linewidth]{figures/intro/intro_yee_grid}
	\caption[A representation of the staggered Yee grid.]{\textbf{A representation of the staggered 3D Yee grid.} For the cell at $(i\Delta x, j\Delta y, k\Delta z)$ for spatial centring of the curl operations, including the locations where all system properties are defined.}
	\label{fig:introyeegrid}
\end{figure}
with electric fields defined at integer time steps and magnetic fields at half-integer time steps. PIC codes can function in one-, two- and three-dimension spatially but always with three-dimensions in velocity.

\subsubsection{Smilei}
Smilei (for Simulating Matter Irradiated by Light at Extreme Intensities) is a modern, collaborative, massively parallel, fully relativistic and open source plasma physics PIC code and the major workhorse for this thesis\footnote{At points benchmarks against the EPOCH and Osiris PIC codes were performed.}. 
Produced at École Polytechnique, Paris \cite{derouillatSmileiCollaborativeOpensource2018}, its recent development was motivated by the recent advancements of multi-petawatt facilities both globally and locally with the completion in 2020 of the 10 PW Apollon laser facility and by the availability of supercomputing power which has `skyrocketed' in recent years \cite{derouillatSmileiCollaborativeOpensource2018}. The 3D \ac{PIC} simulations examined in this thesis demanded the parallelisation of almost 20 \% of the computing resources available on the ARCHER2 supercomputer.

\subsubsection{Reference units}
Given the broad range of magnitudes linked to multi-petawatt and femtosecond laser pulses, solid density plasmas, micrometre wavelengths, and attosecond electron bunches, it becomes significantly more convenient to transform them into dimensionless and normalised values. Smilei operates in such units. This normalisation is not chosen \textit{a priori}, instead results can be scaled by an arbitrary reference angular frequency. This is extremely useful when working with boosted frames of reference. As this thesis focuses on the interaction of a laser pulse with plasma, the laser pulse angular frequency, $\omega_\mathrm{L}$ is set as the frequency of reference. A list of the most common normalisations is provided in Table \ref{tab:intro-normalisations}.

\begin{table}
	\begin{center}
		\begin{tabular}{ccc}
			\hline \hline
			Units of & SI units & Normalisation \\
			\hline
			velocity & \unit{m.s^{-1}} & $c$ \\
			charge & C & $e$ \\
			mass & kg & $m_\mathrm{e}$ \\
			momentum & \unit{kg.m.s^{-1}} & $m_\mathrm{e}c$ \\
			energy/temperature & J & $m_\mathrm{e}c^2$ \\
			time & s & $\omega^{-1}_\mathrm{L}$ \\
			length & m & $c/\omega_\mathrm{L}$ \\
			number density & \unit{m^{-3}} & $n_\mathrm{c}$ \\
			electric field & \unit{V.m^{-1}} & $m_\mathrm{e}c\omega_\mathrm{L}/e$ \\
			\hline \hline
		\end{tabular}
		\caption{\label{tab:intro-normalisations} \textbf{Smilei normalisations for common quantities.} The laser angular frequency $\omega_\mathrm{L}$ is set at the reference angular frequency.}
	\end{center}
\end{table}

\subsubsection{Simulation parameters}\label{sec:intro-general_simulation_paramers}
\textit{Silver-M$\ddot{u}$ller} boundary conditions, chosen for the simulation box edges \cite{barucqAsymptoticBehaviorSolutions1997}, absorb and inject electromagnetic waves and particles. There can be non-physical reflection of electromagnetic waves at such boundaries leading to some error.

The quasi-particle shape function $S(\mathbf{x})$ determines the projection of particle charge onto the grid. It is symmetric in all dimensions with respect to $\mathbf{x}$ and extends over $n$ cells of width $\Delta x$ in each direction where $n$ is the interpolation order. It can be written as a product across $D$ dimensions,
\begin{equation}
	S(\mathbf{x}) = \Pi^D_{\mu = 1}s^{(n)}(x^\mu).
\end{equation}
Smilei implements orders 2, 3 and 4, the explicit shape functions are
\begin{subequations}
	\begin{align}
		s^2 (n) &= \begin{cases}
			\frac{1}{\Delta x}\left(1-\left|\frac{x}{\Delta x}\right| \right)  & \text{if } |x| \le \Delta x, \\
			0  & \text{otherwise,}
		\end{cases} \\
		s^3(n) &= \begin{cases}
			\frac{3}{4\Delta x}\left(1-\frac{4}{3}\left(\frac{x}{\Delta x}\right)^2 \right)  & \text{if } |x| \le \frac{1}{2}\Delta x, \\
			\frac{9}{8\Delta x}\left(1-\frac{2}{3}\left|\frac{x}{\Delta x}\right| \right)^2  & \text{if } \frac{1}{2}\Delta x <|x| \le \frac{3}{2} \Delta x, \\
			0  & \text{otherwise,}
		\end{cases}  \\
		s^4(n) &= \begin{cases}
			\frac{2}{3 \Delta x}\left( 1-\frac{3}{2}\left(\frac{x}{\Delta x}\right)^2 + \frac{3}{4}\left| \frac{x}{\Delta x}\right| ^3 \right)  & \text{if } |x| \le \Delta x, \\
			\frac{4}{3 \Delta x}\left(1-\frac{1}{2}\left|\frac{x}{\Delta x}\right| ^3 \right)  & \text{if } \Delta x <|x| \le 2\Delta x, \\
			0  & \text{otherwise.}
		\end{cases} 
	\end{align}
\end{subequations}

While the correct implementation of collisions in PIC codes remains an open problem \cite{Collisions}, Smilei has implemented relativistic binary collisions between macroparticles using a Monte-Carlo-based scheme \cite{perezImprovedModelingRelativistic2012}. The aforementioned QED processes of Inverse Compton scattering and non-linear Breit-Wheeler pair production are included using the built-in Smilei packages \cite{derouillatSmileiCollaborativeOpensource2018}. These processes can lead to cascades of many particles being added to the simulations. Macroparticle merging can increase simulation efficiency and reduce the memory footprint. Smilei implements such a scheme, inspired by that designed by Vranic \textit{et al} \cite{vranicParticleMergingAlgorithm2015}, that is computationally efficient and conserves energy, momentum and charge within a cell. While Smilei contains modules to handle ionisation, these are deemed unnecessary for the laser intensities considered in this thesis, as highlighted by Figure \ref{fig:intro_lasersystems}.

\subsubsection{Parallelisation in practice}
In the following discussion, where differences in language occur, objects are given their ARCHER2 (Smilei) names. The Smilei simulation box consists of a grid of cells as in Figure \ref{fig:introsmileiparallelisation}.
\begin{figure}
	\centering
	\includegraphics[width=1\linewidth]{figures/intro/intro_smilei_parallelisation}
	\caption[Smilei simulation box decomposition into cells, patches and MPI patch collections.]{\textbf{Smilei simulation box decomposition into cells, patches and MPI patch collections.} Cells are grouped into patches, patches are grouped into MPI patch collections. These collections are assigned patches contiguously along the Hilbert curve.}
	\label{fig:introsmileiparallelisation}
\end{figure}
The box is decomposed into \textit{patches} consisting of many cells. Patches are arranged into \textit{MPI patch collections} assigned contiguously along a Hilbert curve.

Archer consists of many \textit{CPUs (cores)} that can each perform computational tasks. CPUs are grouped into \textit{nodes}. Memory is shared within a node such that all CPUs (cores) in a node can operate on the same data. When optimised, ARCHER2’s memory in each node is split into 8 \textit{sockets}. These 8 sockets each perform a \textit{task (MPI process)}. Each task (MPI process) has 16 CPUs (cores) assigned that each perform a \textit{thread}. A thread is a sequence of instructions from the program.

Each task (MPI process) handles one \textit{MPI patch collection}. Threads work through patches. Figure \ref{fig:introsmileiparallelisationcomplex} represents this division of labour.
\begin{figure}
	\centering
	\includegraphics[width=1\linewidth]{figures/intro/intro_smilei_parallelisation_complex}
	\caption[Representation of the interaction of the ARCHER2 hardware and software components when running Smilei.]{\textbf{Representation of the interaction of the ARCHER2 hardware and software components when running Smilei. Each CPU is assigned a thread, each task is carried out by one socket. A task tackles one MPI patch collection with threads working through patches.}}
		\label{fig:introsmileiparallelisationcomplex}
\end{figure}
Threads do not need to wait for other threads in their process to tackle new patches in their MPI patch collection. This is a form of local \textit{dynamic load balancing}. If an MPI patch collection is overloaded, a patch is offloaded to a contiguous MPI patch collection.

There are several considerations for simulation optimisation. Tasks (MPI processes) should always be assigned more patches than threads. To apply the Hilbert curve, the number of patches in each dimension must be a power of 2. The fewer cells in a patch, the more efficient the load balancing, however, the cost of synchronisation between patches increases, although generally the gain in efficiency from load balancing by increasing patch number will win out at later times with increasing efficiency for increased frequency of load balancing \cite{derouillatSmileiCollaborativeOpensource2018}. Note that smaller patches are preferable when there are small regions with large numbers of particles, as in laser-solid surface interactions, however, the minimum patch size is dependent on the shape function of the macro-particles.

\subsubsection{Sources of error inherent to PIC codes}
Despite their relatively intuitive interpretation, PIC codes are famously finickety and prone to errors, most notably that of numerical self heating\footnote{Standard PIC code algorithms are charge and momentum conserving but not energy conserving \cite{derouillatSmileiCollaborativeOpensource2018}.}.

To ensure stability, or at least to minimise instability, there are several conditions which must be met. Naturally, the time step, $\Delta t$, and cell size ($\Delta x$x$\Delta y$x$\Delta z$) must adequately capture all interesting features of a given simulation. Typically such features are plasma wave oscillations,
\begin{equation}
	\Delta t \omega_\mathrm{p} \ll 1,
\end{equation} 
and laser pulse electromagnetic field oscillations or higher order harmonics of the laser pulse if that is of interest, for the $n$th harmonic
\begin{equation}
	\Delta t\omega_\mathrm{L}n \ll 1.
\end{equation} 
Note this also ensures that macroparticles are smaller than the wavelengths of the system, a requirement to ensure they mimic real particles \cite{okudaCollisionsPlasmaFiniteSize1970}. Relativistic PIC codes must satisfy the much-acclaimed \cite{demouraCourantFriedrichsLewy2013} Courant-Friedrichs-Lewy condition,
\begin{equation}
	\frac{1}{c\Delta t} > \sqrt{\frac{1}{(\Delta x)^2}+\frac{1}{(\Delta y)^2}+\frac{1}{(\Delta z)^2}},
\end{equation}
thus preventing light and relativistic particles from crossing a cell in one timestep and generating numerical Cerenkov radiation \cite{birdsall2004plasma}. As with real plasmas and real particles, to avoid numerical charge fluctuation and ensure the collective behaviour of macroparticles,
\begin{equation}
	\frac{4}{3}\pi\lambda_\mathrm{D}^3n_\mathrm{macro} = N_\mathrm{D,macro} \ggg 1,
\end{equation}
where $n_\mathrm{macro}$ is the macroparticle density \cite{birdsall2004plasma}. To avoid numerical heating, the cell size must resolve the Debye length,
\begin{equation}
	\frac{\lambda_\mathrm{D}}{\Delta x} \ge 1,
\end{equation}
failure to do so may cause plasma self-heating until the Debye length matches the cell size \cite{birdsall2004plasma}. Interestingly, Brackbill \textit{et al} \cite{brackbillEnergyMomentumConservation2016} observed in their simulations that setting $\lambda_\mathrm{D}/\Delta x =1$ was most effective at reducing spurious heating. Arber \textit{et al} \cite{arberContemporaryParticleincellApproach2015} performed extensive simulations exploring this instability. If the Debye length is not resolved, after an initial period of rapid self-heating, the temperature increases linearly and can be modelled as
\begin{equation}\label{eq:intro-selfheating}
	\frac{\mathrm{d} T_\mathrm{eV}}{\mathrm{d}t_\mathrm{ps}} = \alpha_\mathrm{H} \frac{n^{3/2}_{23} \Delta x^2_\mathrm{nm}}{N_\mathrm{ppc}},
\end{equation}
where $T_\mathrm{eV}$ is the temperature in electron volts, $t_\mathrm{ps}$ is the time in picoseconds, $n$ is the plasma electron number density in units of $10^{23}$ \unit{cm^{-3}}, $\Delta x_\mathrm{nm}$ is the cell size in nanometres and $N_\mathrm{ppc}$ is the number of macroparticles per cell with $\alpha_\mathrm{H}$, the heating coefficient, determined from simulation. For a top-hat macroparticle shape function, they observed $\alpha_\mathrm{H} = 3000$ with an order of magnitude reduction for every increase in order of the shape function. Further improvements were also noted from the use of current smoothing techniques. Note that the heating curves are roughly self-similar at all points and thus while Equation \ref{eq:intro-selfheating} was established in the linear regime only, its scalings remain useful for simulation comparison at all times.

The final instability that shall be discussed is the \textit{finite grid instability}. This is the aliasing error associated with particle properties being deposited at grid points. Is most catastrophic for cold drifting plasmas and depends on the \textit{beam Debye length},
\begin{equation}
	B = \frac{u}{\omega_\mathrm{p} \Delta x},
\end{equation}
where $u$ is the beam speed. While their theory predicts stability for $B > 0.25$, Brackbill \textit{et al} \cite{brackbillEnergyMomentumConservation2016} observed instability growth for all beam temperatures in their simulations, although they found the percentage error is a small fraction for $B>10$.

\subsection{Hydrodynamic codes}
A hydrodynamic code approximates the plasma distribution as a fluid. This is achieved by taking the velocity moments of the distribution function, $f_s$ for fluid species $s$. Similar to the charge and current density calculations of Section \ref{sec:ch1-vlasov}, the first three return the number density of particles
\begin{equation}
	n_s = \int f_s \mathrm{d}\mathbf{v},
\end{equation}
the fluid velocity, $\mathbf{u}_s$ and momentum,
\begin{equation}
	\mathbf{p_s} = m_sn_s\mathbf{u}_s = \int f_s \mathbf{v}\mathrm{d}\mathbf{v},
\end{equation}
and the pressure tensor
\begin{equation}
	P_s = m_s\int f_s (\mathbf{v}-\mathbf{u}_s)(\mathbf{v}-\mathbf{u}_s)\mathrm{d}\mathbf{v}.
\end{equation}
The fluid equations that govern the evolution of these quantities can be extracted from the Vlasov equation, Equation \ref{eq:intro-vlasov} by taking the appropriate moments \footnote{Of course, these equations can also be derived by first principles by considering the dynamics of a fluid element \cite{chenIntroductionPlasmaPhysics2016}}. Likewise  producing the continuity equation
\begin{equation}
	\frac{\partial n_s}{\partial t} + \nabla \cdot (n_s \mathbf{u}_s) = 0,
\end{equation}
for conservation of particle number and the equation of motion
\begin{equation}\label{eq:intro_eom_fluid}
	m_sn_s\left(\frac{\partial \mathbf{u}_s}{\partial t} + (\mathbf{u}_s\cdot \nabla) \mathbf{u}_s \right) = q_sn_s(\mathbf{E}+\mathbf{u}_s \times \mathbf{B}) - \nabla \cdot P_s
\end{equation}
for conservation of momentum. This equation is analogous to the Navier-Stokes equation for standard fluids except for the addition of the electromagnetic fields. The viscosity term is collected within the pressure tensor \cite{chenIntroductionPlasmaPhysics2016}.

Here lies a problem. The presence of the pressure tensor in Equation \ref{eq:intro_eom_fluid} prevents the closure of the system of equations. Taking the second moment of the Vlasov equation to obtain the energy transport equation would then contain the third moment of the distribution function, presenting the same issue. Instead, a \textit{closure relation} is introduced, enabling the construction of a self-contained theory using a finite number of moments.

HYADES is a 1D three fluid (electrons, ions and radiation) radiative hydrodynamic code \cite{larsenHYADESPlasmaHydrodynamics1994} and is the code of choice in this thesis. The equations of mass and energy transport are solved in a Lagrangian coordinate system, \textit{i.e.}, unlike strictly Eulerian PIC codes, the mesh defining regions of the simulation moves with the plasma it describes. To complete the system up to the second moment, the \textit{equation of state}, describing pressure in terms of the temperature and density of the system, and the heat transfer equation are added. 

%TODO justification for the Vlasov equation
% This are some good notes on this http://www.pmaweb.caltech.edu/Courses/ph136/yr2012/1222.3.K.pdf
% Also see here https://www2.mps.mpg.de/solar-system-school/lectures/space_plasma_physics_2007/Lecture_6.pdf
% Actual full derivation is given here: https://www.ucolick.org/~enrico/ast111/material_files/kinetic.pdf

The reduction of dimensionality achieved through this approximation dramatically reduces the computational cost of the simulation. However, much of this thesis relies on the precise velocity distribution of particles. Naturally, such a code cannot model these phenomena and the primary tool is the PIC code.

% There are some additional equations on HYADES in that 1994 paper that might be useful to include

\section{Generating the interaction}
Two experiments are discussed in this thesis, one on ORION, and one planned on GEMINI. Their typical beam parameters are presented in Table \ref{tab:laser_params}.
\begin{table}[]
	\centering
	\begin{tabular}{lccc}
		\hline \hline
		Parameter                & \multicolumn{2}{c}{ORION} & GEMINI \\ 
		Beamlines                & SP1         & SP2         & N \& S  \\ \hline
		Power (PW)               & 0.5         & 1           & 0.5    \\
		Energy (J)               & 200         & 500         & 12     \\
		Wavelength (nm)          & 527         & 1053        & 800    \\
		Parabola, $f/\#$ & $f/3$        & $f/3$        & $f/2$      \\
		Focal spot FWHM ($\mu$m) & < 20        & < 10        & 2      \\
		Duration (fs)            & 500         & 500         & 40     \\
		Shot rate                & 5/day       & 5/day       & 3/min  \\
		Peak $a_0$ (approx)      & 10          & 30          & 24    \\ \hline \hline
	\end{tabular}
	\caption{\textbf{The ORION and GEMINI petawatt class short pulse beamlines for comparison.} The GEMINI North (N) and South (S) beamlines are equivalent.}
	\label{tab:laser_params}
\end{table}
The GEMINI-PW laser facility of the \ac{CLF} at Harwell Campus, Oxfordshire is a petawatt class facility consisting of two 30 fs beams, North and South, each delivering a maximum focused intensity of \qty{2e21}{W.cm^{-2}} at a repetition rate of 0.05 Hz. The commissioning of such high-frequency facilities has ushered in a paradigm shift in high-power laser physics experimentation with the entry of statistically significant results.

The ORION laser facility of AWE, Aldermaston is the UK's most powerful laser system with two short pulse beamlines, SP1 and SP2 and ten `long pulse' beams. The ORION SP1 beamline is produced by passing the SP2 beamline through two frequency doubling crystals creating two equivalent beamlets that are then superimposed at the target. The double beamlet structure at the near-field is imaged in Figure \ref{fig:oriondoublebeamletsnearfield}.
\begin{figure}
	\centering
	\includegraphics[width=0.5\linewidth]{figures/orion/orion_doublebeamlets_near_field}
	\caption[Image of the ORION SP1 double beamlet structure in the near-field.]{\textbf{Image of the ORION SP1 double beamlet structure in the near-field.} The two beamlets are then superimposed on the target.}
	\label{fig:oriondoublebeamletsnearfield}
\end{figure}

Laser contrast is a major concern for the interaction discussed in this thesis. There is always some spurious signal arriving before the main pulse of interest. In high-power laser systems, this \textit{prepulse} can be itself of ionising intensities, dramatically altering the initial conditions of the system. Generally, the prepulse preheats the plasma leading to expansion. This is catastrophic if one's aim is to interact a laser with a solid density plasma.

The ORION SP2 beamline temporal intensity envelope can be modelled as
\begin{equation}\label{eq:orion_SP2_temporal}
	I_\mathrm{SP2} \sim (I_{0 \ \mathrm{p}}\sech(t/t_\mathrm{p})^2 + I_{0 \ \mathrm{pp}}e^{-(t/t_\mathrm{pp})^2} + I_{0 \ \mathrm{ps}}e^{-\mathrm{abs}(t)/t_\mathrm{ps}} + I_{0 \ \mathrm{ns}}e^{-(t/t_\mathrm{ns})^8}),
\end{equation}
where the constants are detailed in Table \ref{tab:orion_pedestals} \cite{dhillierModelORIONContrast22}.
\begin{table}[]
	\centering
	\begin{tabular}{lcc}
		\hline\hline
		Pedestal, $i$                & $I_{0 \ i}$ & $t_i$  \\ \hline
		Main pulse, p                & 1           & 0.2 ps \\
		Picosecond pump residual, pp & $10^{-3}$   & 3 ps   \\
		Picosecond pedestal, ps      & \num{5e-5}  & 8 ps   \\
		Nanosecond pedestal, ns      & $10^{-11}$  & 3 ns  \\ \hline \hline
	\end{tabular}
	\caption{\textbf{The native ORION SP2 pulse and prepulse pedestal constants as defined in Equation \ref{eq:orion_SP2_temporal}.}}
	\label{tab:orion_pedestals}
\end{table}
The picosecond pump residual arises from parametric fluorescence in the ps OPA, the picosecond pedestal from scatter in the stretchers and noise on the OPA pump laser and the nanosecond pedestal from parametric fluorescence from the nanosecond OPAs. The `long' duration of the prepulse precludes the use of PIC codes. It is here that the hydrodynamic codes can be more appropriate.

The SP1 temporal intensity profile is
\begin{equation}
	I_\mathrm{SP1} \sim (I_\mathrm{SP2}^2 + I_\mathrm{SP2} \times 10^{-8}).
\end{equation}
The second term arises from limitations in the harmonic separation system. The frequency doubling mechanism is not 100 \% efficient, \textit{i.e.} some of SP2 beamline remains and must be filtered out. The main pulse of the GEMINI beamlines is more appropriately modelled as a Gaussian.

A \ac{PM} is an effective and now standardised tool for the improvement of laser contrast. These single-use optics have initially an \ac{AR} coating of reflectivity $R_1$. As the incident laser fluence passes the damage threshold of the optic, reflecting plasma forms on the front surface. The optic is `switched on' with a reflectivity $R_2$. The \ac{PM} therefore improves the contrast by a factor $R_2/R_1$. There is a balancing act in choosing the peak intensity on the \ac{PM}. Too early and the PM will switch on before the arrival of the main pulse. Too low and there is not enough plasma at the surface at the arrival of the main pulse and $R_2$ is lowered. A peak intensity around \qty{1e16}{W.cm^{-2}} is best \cite{caiTimeresolvedMeasurementsReflectivity2009}. PMs also act as a low-pass spatial filter, improving the quality of the focal spot on target \cite{doumyCompleteCharacterizationPlasma2004}. The use of PMs before the main target ensure the required conditions for a relativistic laser pulse to produce a \ac{RPM}, the main focus of this thesis.

\section{Summary}
This chapter has introduced the field of HED physics. The rapid advancements towards the exascale in both supercomputing and laser power have been discussed. The basic equations that govern the system have been supplied and some simple derivations performed to describe the dynamics of particles in the classical, relativistic and QED regimes. Details have been given of two suitable simulation code types, PIC and hydrodynamic, and of laser systems that enable the modelling and generation respectively of the interaction of interest in this thesis. Some of the potential pitfalls and corresponding mitigations for each have also been highlighted. Thus, it is now time to introduce that interaction.